% Created 2020-10-19 Mon 02:04
% Intended LaTeX compiler: lualatex
\documentclass[11pt]{article}
\usepackage{graphicx}
\usepackage{grffile}
\usepackage{longtable}
\usepackage{wrapfig}
\usepackage{rotating}
\usepackage[normalem]{ulem}
\usepackage{amsmath}
\usepackage{textcomp}
\usepackage{amssymb}
\usepackage{capt-of}
\usepackage{hyperref}
\usepackage{tabularx}
\usepackage{etoolbox}
\makeatletter
\def\dontdofcolorbox{\renewcommand\fcolorbox[4][]{##4}}
\AtBeginEnvironment{minted}{\dontdofcolorbox}
\makeatother
\usepackage[newfloat]{minted}
\author{Mark Armstrong}
\date{October 19th, 2020}
\title{Computer Science 3MI3 – Scala tidbits}
\hypersetup{
   pdfauthor={Mark Armstrong},
   pdftitle={Computer Science 3MI3 – Scala tidbits},
   pdfkeywords={},
   pdfsubject={Examples and code tidbits which help us understand Scala and functional programming in general.},
   pdfcreator={Emacs 27.0.91 (Org mode 9.4)},
   pdflang={English},
   colorlinks,
   linkcolor=blue,
   citecolor=blue,
   urlcolor=blue
   }
\begin{document}

\maketitle
\tableofcontents


\section*{Example of \texttt{Option[Either[A,B]]} recursive functions}
\label{sec:org76f710d}
In the first assignment, the final interpreter requires
the return type of \texttt{Option[Either[Int,Boolean]]}.
The idea is that \texttt{Option[Either[Int,Boolean]]} contains
\begin{itemize}
\item the value \texttt{None} which represents failure,
\item for each integer \texttt{n}, the value \texttt{Some(Left(n))},
\item for each boolean \texttt{b}, the value \texttt{Some(Right(b))}
\end{itemize}
so this return type can be used for an interpreter
which might return an integer, might return a boolean, or
might fail.

This type can be somewhat awkward to work with,
at least until we discuss the concept of \emph{combining} such values
(also known as \emph{monadic bind}'s.)

For the moment, this example code may help you with
your own definitions for the assignment.

Here, we work with that same type of \texttt{Option[Either[Int,Boolean]]} to
define \texttt{sum} and \texttt{product} methods, with the idea that
\begin{itemize}
\item the \texttt{sum} of two integers is their sum in the usual sense,
\item the \texttt{sum} of two booleans is the result of “logically or-ing” them,
\item and the \texttt{sum} of two values of different types is undefined.
\end{itemize}
The \texttt{product} is similar, but uses logical and for the boolean case.

The important takeaway here is how we (using pattern matching)
\begin{enumerate}
\item “unwrap” the argument values \texttt{x} and \texttt{y},
\item then in the cases we have two integers or two booleans,
perform the relevant operation before
\item “rewrap”-ing the result; and
\item for the other cases,
\begin{itemize}
\item (having a \texttt{None} for one or both of the arguments,
\item or having an integer for one argument and a boolean for the other)
\end{itemize}
which we summarise with the “catch-all” case using underscores,
we return \texttt{None}, representing failure.
\end{enumerate}
\begin{minted}[breaklines=true]{scala}
def sum(x: Option[Either[Int,Boolean]],
        y: Option[Either[Int,Boolean]]): Option[Either[Int,Boolean]] =
  (x, y) match {
    case (Some(Left(m)), Some(Left(n))) => Some(Left(m + n))
    case (Some(Right(a)), Some(Right(b))) => Some(Right(a || b))
    case (_ , _) => None
  }

def product(x: Option[Either[Int,Boolean]],
            y: Option[Either[Int,Boolean]]): Option[Either[Int,Boolean]] =
  (x, y) match {
    case (Some(Left(m)), Some(Left(n))) => Some(Left(m * n))
    case (Some(Right(a)), Some(Right(b))) => Some(Right(a && b))
    case (_ , _) => None
  }
\end{minted}

You may have noticed the extreme similarity between these two methods.
And indeed, you may see similarity between these methods
and the code you may write for the assignment.

If you are considering how we can \emph{generalise} these
operations in order to avoid repeating ourselves as I have above
(having rewritten or copied/pasted the pattern matching,
changing on the first two righthand sides)
stay tuned; as mentioned earlier, we will discuss these concepts
later on in the course.
\end{document}
