% Created 2020-09-24 Thu 18:30
% Intended LaTeX compiler: lualatex
\documentclass[11pt]{article}
\usepackage{graphicx}
\usepackage{grffile}
\usepackage{longtable}
\usepackage{wrapfig}
\usepackage{rotating}
\usepackage[normalem]{ulem}
\usepackage{amsmath}
\usepackage{textcomp}
\usepackage{amssymb}
\usepackage{capt-of}
\usepackage{hyperref}
\usepackage{tabularx}
\usepackage{etoolbox}
\makeatletter
\def\dontdofcolorbox{\renewcommand\fcolorbox[4][]{##4}}
\AtBeginEnvironment{minted}{\dontdofcolorbox}
\makeatother
\usepackage[newfloat]{minted}
\author{Mark Armstrong}
\date{September 24th, 2020}
\title{Computer Science 3MI3 – Prolog tidbits}
\hypersetup{
   pdfauthor={Mark Armstrong},
   pdftitle={Computer Science 3MI3 – Prolog tidbits},
   pdfkeywords={},
   pdfsubject={Examples and code tidbits which help us understand Prolog},
   pdfcreator={Emacs 27.0.90 (Org mode 9.3.8)},
   pdflang={English},
   colorlinks,
   linkcolor=blue,
   citecolor=blue,
   urlcolor=blue
   }
\begin{document}

\maketitle
\tableofcontents


\section*{Unguarded cases}
\label{sec:org81cb9b9}
Not “guarding” the cases in your predicate definitions
can result in slightly different behaviour when it comes to backtracking.
\begin{minted}[breaklines=true]{prolog}
% This tidbit illustrates the effect of
% not correctly restricting the application of different cases
% in our predicate definitions.


% This predicate is well defined; two cases,
% each of which is restricted to a particular form of list.
fact1([]) :- writeln('fact1, base case'), true, !.
fact1([_|T]) :- writeln('fact2, recursive case'),fact1(T).



% This similar predicate includes an extra case.
fact2([]) :- writeln('fact2, base case'), true, !.
fact2([_|T]) :- writeln('fact2, recursive case'), fact2(T).
% The last case seems unreachable.
% But this will make the search continue and return `false`
% as a second result. (Assuming the querier presses ; or <space>.)
fact2(_) :- writeln('fact2, supposedly unreachable case'), false.
\end{minted}

You can try out this example by loading the \href{./src/unguarded-cases.pl}{source code} into your REPL.
\begin{minted}[breaklines=true]{prolog}
% Query test to see how Prolog will pause after the query,
% allowing you to to use ; or <space> to then get `false.`
test :- fact2([1]).



% Querying expected will show what we would expect to happen instead;
% the search returns `true.` and halts, not allowing `;` or `<space>`.
expected :- fact1([1]).
\end{minted}
\end{document}
