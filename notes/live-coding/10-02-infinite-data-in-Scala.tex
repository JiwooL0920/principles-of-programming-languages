% Created 2020-10-07 Wed 03:30
% Intended LaTeX compiler: lualatex
\documentclass[11pt]{article}
\usepackage{graphicx}
\usepackage{grffile}
\usepackage{longtable}
\usepackage{wrapfig}
\usepackage{rotating}
\usepackage[normalem]{ulem}
\usepackage{amsmath}
\usepackage{textcomp}
\usepackage{amssymb}
\usepackage{capt-of}
\usepackage{hyperref}
\usepackage{tabularx}
\usepackage{etoolbox}
\makeatletter
\def\dontdofcolorbox{\renewcommand\fcolorbox[4][]{##4}}
\AtBeginEnvironment{minted}{\dontdofcolorbox}
\makeatother
\usepackage[newfloat]{minted}
\usepackage{unicode-math}
\usepackage{unicode}
\author{Mark Armstrong}
\date{\today}
\title{Infinite data in Scala}
\hypersetup{
   pdfauthor={Mark Armstrong},
   pdftitle={Infinite data in Scala},
   pdfkeywords={},
   pdfsubject={We discuss the notion of “infinite data”, and show how we can represent such data in Scala.},
   pdfcreator={Emacs 27.0.90 (Org mode 9.4)},
   pdflang={English},
   colorlinks,
   linkcolor=blue,
   citecolor=blue,
   urlcolor=blue
   }
\begin{document}

\maketitle
\tableofcontents


\section{Introduction}
\label{sec:orgb12fc01}
These notes were created for, and in some parts \textbf{during},
the lecture on October 2nd and the following tutorials.

\section{Motivation}
\label{sec:org907a17b}
In our current lectures, we have been discussing the λ-calculus,
and as of the time of writing this, we are about to discuss
reduction strategies.

The reduction strategies used in the λ-calculus
correspond to the parameter-passing methods
used in conventional programming languages.

\section{Call by value and call by name}
\label{sec:org9d3deb5}
Today, we are interested in the
“call by value” and “call by name”
reduction strategies/parameter-passing methods.
\begin{itemize}
\item Under “call by value” semantics,
a function application is only evaluated after
its arguments have been reduced to values.
\item Under “call by name” (or “call by need”) semantics,
no arguments in a function application are evaluated before
the function is applied.
\begin{itemize}
\item (And no reduction is allowed inside abstractions.)
\end{itemize}
\end{itemize}

(Under both schemes, the leftmost, outermost valid reduction is done first.)

We can see the difference by considering a sample redex
in the λ-calculus.
\begin{minted}[breaklines=true]{text}
(λ x → x x)((λ y → y) (λ z → z))
\end{minted}
Notice that there are two possible applications to carry out;
applying \texttt{λ z → z} to \texttt{λ y → y}, or
applying \texttt{(λ y → y) (λ z → z)} to \texttt{λ x → x x}.

A call-by-value semantics requires that the right side be evaluated first;
“a function application is only evaluated
after its arguments have been reduced to values”.
So we cannot perform the outermost application until the term on the right
is reduced.
\begin{minted}[breaklines=true]{text}
  (λ x → x x)((λ y → y) (λ z → z))
⟶ (λ x → x x)(λ z → z)
⟶ (λ z → z) (λ z → z)
→ λ z → z
= id
\end{minted}

A call-by-name semantics instead requires that the outside
application is evaluated first;
“no arguments in a function application are evaluated before
the function is applied”.
So we cannot perform the application in the term on the right
until we apply the outermost application.
\begin{minted}[breaklines=true]{text}
  (λ x → x x)((λ y → y) (λ z → z))
⟶ ((λ y → y) (λ z → z)) ((λ y → y) (λ z → z))
→ (λ z → z) ((λ y → y) (λ z → z))
→ (λ y → y) (λ z → z)
→ λ z → z
= id
\end{minted}

\section{Parameter passing in Scala}
\label{sec:org3b110a2}
By default, Scala uses a call-by-value strategy.
\begin{minted}[breaklines=true]{scala}
def f(x: Int): Unit = { println("Called f with argument"); println(x) }
\end{minted}

You may \emph{opt-in} to call by name using what they call “by name parameters”;
simply prepend \texttt{=>} to the type.
\begin{minted}[breaklines=true]{scala}
def g(y: => Int) : Unit = { println("Called g with argument"); println(y) }
\end{minted}

Let us create a value that is not immediately evaluated,
and which communicates when it is evaluated,
so we can use it as an argument to test out our above definitions.
One way to do this is by defining it as a method with no parameters.
\begin{minted}[breaklines=true]{scala}
def x: Int = { println("Evaluated x"); 1 }
\end{minted}
:TODO: explain the term lazy

Scala also has “lazy” values, which are not evaluated until used.
:TODO: how is this different than x?
\begin{minted}[breaklines=true]{scala}
lazy val y: Int = { println("Evaluated y"); 2 }
\end{minted}

\section{What's the point?}
\label{sec:orgc80bf2e}
Why do we want to be allowed to use call by name semantics
and lazy values?

It may make some tasks easier conceptually;
for instance, one common use case involves a function
on the natural numbers, such as one that returns the “nth” prime.

We could certainly write such a function,
but an alternative approach is to \emph{lazily} construct
the list of all primes by \emph{filtering} the list of all naturals.
Since it is lazily constructed, no space is used
until we begin to look up elements in the list.
(Code courtesy of this \href{https://stackoverflow.com/q/15594227/2041536}{StackOverflow} post,
modified to work with \texttt{LazyList}.)
\begin{minted}[breaklines=true]{scala}
lazy val ps: LazyList[Int] =
  2 #:: LazyList.from(3).filter(i => ps.takeWhile{j => j * j <= i}.forall{ k => i % k > 0});
\end{minted}

And, because lazy data is not re-evaluated,
once we have looked up elements, the already calculated portion
of the list is automatically cached for us!
This is (in some instances) an advantage over the function.

Now, we have subtlely used another concept enabled by lazy values
and call-by-name semantics in the above: an infinite list!

\section{Infinite data!}
\label{sec:orgf4b1597}
\subsection{Not with lists…}
\label{sec:org636e83c}
When we discuss lists in computer science,
they are usually defined as having \emph{finite} length.

Let us try to break away from that convention.
We can define an infinite list by using recursion.
But what does you intuition tell you will happen here?
\begin{minted}[breaklines=true]{scala}
lazy val ones: List[Int] = 1 :: ones
\end{minted}

\subsection{…instead, with lazy lists or streams!}
\label{sec:org912c4f8}
The \href{https://www.scala-lang.org/api/current/scala/collection/immutable/LazyList.html}{lazy list} type and
the \href{https://www.scala-lang.org/api/current/scala/collection/immutable/Stream.html}{stream} type (now deprecated in favour of lazy lists)
actually allow us to define such lists.
\begin{minted}[breaklines=true]{scala}
lazy val ones: LazyList[Int] = 1 #:: ones
\end{minted}

We can then make these lists more interesting
by filtering, zipping, etc.;
or by writing more interesting recursive definitions.

\section{\sout{(Not)} Implementing our own infinite datatypes}
\label{sec:orgd958a98}
\subsection{The approach}
\label{sec:org75ac327}
To implement such a type \texttt{T} with the algebraic datatype approach
we have been using, we may use (recursive) parameters
of the form \texttt{Unit => T}; that is,
the type of \emph{functions} from \texttt{Unit} to \texttt{T}.
(Recall that \texttt{Unit} is the type with just one value,
written \texttt{()} in Scala.)

Function parameters are \emph{never} evaluated until
the function is invoked (called.)
\textbf{This will approach will work in any language with
higher-order functions.}

For example, we can define our own variant of streams.
\begin{minted}[breaklines=true]{scala}
sealed trait Stream[+A] // Stream is covariant (marked by the +)
case object SNil extends Stream[Nothing] // The singleton Nil object
case class Cons[A](a: A, f: Unit => Stream[A]) extends Stream[A]
\end{minted}

\subsection{Dealing with the tail function}
\label{sec:orge8a4f47}
Given a stream of the form \texttt{Cons(a,f)},
the \texttt{f} parameter does not directly give us the “tail” stream;
we must invoke it, writing \texttt{f()}, to obtain the tail.

In the other direction,
in order to construct a stream of the form \texttt{Cons(a,f)},
we must “wrap” the tail stream in a function definition.
Rather than writing out a \texttt{def} and giving this function a name
(which in Scala lingo would make it a method),
we can use an \emph{anonymous} definition,
such as below.
\begin{minted}[breaklines=true]{scala}
val ones: Stream[Int] = Cons(1, _ => ones)
\end{minted}
The \texttt{\_ => ones} indicates a function of one argument;
the \texttt{\_} is used in place of a variable name,
since we are not using the variable
(because it is of type \texttt{Unit}, there is really no use in
giving it a name; it must be \texttt{()}.)

\begin{center}
So, to construct a stream, we must “wrap” the tail;
this is also called \textbf{delaying} the tail.

To deconstruct (stream) a stream, we must “invoke” the tail;
this is also called \textbf{forcing} the tail.

\textbf{The concepts of delaying and forcing data are generally applicable;}
\textbf{we may later even consider a type which is just (potentially)}
\textbf{delayed data.}
\end{center}

\subsection{\texttt{case object}?}
\label{sec:org24d4b7d}
We adopt here and now the convention of making our base case
a \texttt{case object}, rather than a \texttt{case class}.
The difference is that there is exactly one instance of
a \texttt{case object} (it is a singleton),
whereas there can be many of a \texttt{case class}.
Since there can only be one instance of \texttt{SNil},
this instance needs to be a member of \texttt{Stream[A]} for any \texttt{A},
so it must be a member of \texttt{Stream[Nothing]}, since \texttt{Nothing} is
the only subtype of all types.
This does require us to mark \texttt{Stream} as \emph{covariant},
meaning that \texttt{Stream[A]} is a subtype of \texttt{Stream[B]} if
\texttt{A} is a subtype of \texttt{B}.
(We will discuss subtyping, variance and covariance
in more detail later in the course.)

\subsection{Defining some methods on our streams}
\label{sec:org2ade41c}
We need some methods to allow us to do
even basic tasks with our new Stream type.

For instance, we need some convenient ways
to define streams.
For instance, we can construct constant streams
that just repeat the same value.
\begin{minted}[breaklines=true]{scala}
def constantStream[A](a: A): Stream[A] =
  Cons(a, _ => constantStream(a))
\end{minted}

But it's relatively hard to work with streams
when we can only see the first value
when they are printed.
So, we pause giving ways to construct streams
in order to give a way to \emph{deconstruct} them.
This method converts the first number of elements
into a list which can be nicely and easily displayed.
\begin{minted}[breaklines=true]{scala}
def take[A](n: Int, s: => Stream[A]): List[A] = s match {
  case SNil => Nil
  case Cons(a,f) => n match {
    case n if n > 0 => a :: take(n-1,f())
    case _ => Nil
    }
  }
\end{minted}

Now, let's define a method that will make it easier
to define some other ways of constructing streams:
prepending a finite number of elements from a list
to a stream.
\begin{minted}[breaklines=true]{scala}
def prepend[A](l: List[A], s: => Stream[A]): Stream[A] = l match {
    case Nil => s
    case (h :: t) => Cons(h, _ => prepend(t, s))
  }
\end{minted}

\subsection{Take care: \textbf{make arguments of type \texttt{Stream} by name arguments}}
\label{sec:orgfa81477}
Notice that in the definition of \texttt{prepend},
which is our first method to take a stream as an argument,
we marked the stream as a “by name” argument,
making it lazily evaluated.

This is because if we try to
use \texttt{prepend} to define a stream \emph{recursively},
so the recursive call is inserted as an argument
to \texttt{prepend} (e.g., \texttt{val ones: Stream[Int] = repeat(List(1), ones)}),
if that argument were \emph{not} by name, it would be evaluated,
and we would get a stack overflow due to unbounded recursion
(this occurred during the lecture on Monday, October 5th.)

\begin{center}
\textbf{In general, mark your arguments of type \texttt{Stream} by name,}
\textbf{in order to avoid accidental evaluation.}
\end{center}

\subsection{More methods on our streams}
\label{sec:orge6d26f3}
Using \texttt{prepend}, let us define two ways
to go from a stream to a list;
either just convert the list to a finite stream
(one that actually does end with \texttt{SNil} eventually),
or “repeat” the list infinitely.
\begin{minted}[breaklines=true]{scala}
def toStream[A](l: List[A]): Stream[A] = prepend(l, SNil)

def repeat[A](l: List[A]): Stream[A] = prepend(l, repeat(l))
\end{minted}

We can also \emph{append} a list to a stream, or more appropriately,
append one stream to another.
You might ask here though,
“what happens if the stream is infinite”?
The answer: nothing! The append will just never take place,
since we never reach the \texttt{SNil} case while moving
though the first stream.
\begin{minted}[breaklines=true]{scala}
def append[A](s: => Stream[A], t: => Stream[A]): Stream[A] = s match {
    case SNil => t
    case Cons(a, f) => Cons(a, _ => append(f(),t))
  }
\end{minted}
\end{document}
