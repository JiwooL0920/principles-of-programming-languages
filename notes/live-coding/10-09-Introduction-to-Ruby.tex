% Created 2020-10-09 Fri 13:34
% Intended LaTeX compiler: lualatex
\documentclass[11pt]{article}
\usepackage{graphicx}
\usepackage{grffile}
\usepackage{longtable}
\usepackage{wrapfig}
\usepackage{rotating}
\usepackage[normalem]{ulem}
\usepackage{amsmath}
\usepackage{textcomp}
\usepackage{amssymb}
\usepackage{capt-of}
\usepackage{hyperref}
\usepackage{tabularx}
\usepackage{etoolbox}
\makeatletter
\def\dontdofcolorbox{\renewcommand\fcolorbox[4][]{##4}}
\AtBeginEnvironment{minted}{\dontdofcolorbox}
\makeatother
\usepackage[newfloat]{minted}
\usepackage{unicode-math}
\usepackage{unicode}
\author{Mark Armstrong}
\date{\today}
\title{Introduction to Ruby}
\hypersetup{
   pdfauthor={Mark Armstrong},
   pdftitle={Introduction to Ruby},
   pdfkeywords={},
   pdfsubject={A brief tour of Ruby.},
   pdfcreator={Emacs 27.0.91 (Org mode 9.4)},
   pdflang={English},
   colorlinks,
   linkcolor=blue,
   citecolor=blue,
   urlcolor=blue
   }
\begin{document}

\maketitle
\tableofcontents


\section{Introduction}
\label{sec:org4305229}
These notes were created for, and in some parts \textbf{during},
the lecture on October 9th and the following tutorials.

\section{Motivation}
\label{sec:org277e45a}
So far, we have been investigating
\begin{itemize}
\item the functional paradigm, using Scala,
\begin{itemize}
\item which happens to also be a pure object-oriented language, and
\end{itemize}
\item the logical paradigm, using Prolog.
\end{itemize}

We now investigate an \emph{imperative} pure object-oriented language, Ruby.

Ruby's syntax is also heavily influenced by Lisp,
and the final language we will investigate later in the course
is a Lisp.
So Ruby acts as a stepping stone to that point.

\section{Background on Ruby}
\label{sec:org98679b3}
Ruby is an almost purely object-oriented language,
heavily inspired by Smalltalk, Lisp and Perl.

It's creator, Yukihiro “Matz” Matsumoto,
has documented these inspirations in
\href{http://blade.nagaokaut.ac.jp/cgi-bin/scat.rb/ruby/ruby-talk/179642}{a post on ruby-talk}.
\begin{quote}
Ruby is a language designed in the following steps:

\begin{itemize}
\item take a simple lisp language (like one prior to CL).
\item remove macros, s-expression.
\item add simple object system (much simpler than CLOS).
\item add blocks, inspired by higher order functions.
\item add methods found in Smalltalk.
\item add functionality found in Perl (in OO way).
\end{itemize}
\end{quote}

Ruby was a language born out of Matz's interests
and love of certain language features.
He's \href{https://ruby-doc.org/docs/ruby-doc-bundle/FAQ/FAQ.html}{said}
about its creation
\begin{quote}
Well, Ruby was born on February 24 1993.
I was talking with my colleague about the possibility of
an object-oriented scripting language.
I knew Perl (Perl4, not Perl5), but I didn't like it really,
because it had smell of toy language (it still has).
The object-oriented scripting language seemed very promising.

I knew Python then. But I didn't like it,
because I didn't think it was a true object-oriented language
─ OO features appeared to be add-on to the language.
As a language manic and OO fan for 15 years,
I really wanted a genuine object-oriented,
easy-to-use scripting language.
I looked for, but couldn't find one.

So, I decided to make it.
It took several months to make the interpreter run.
I put it the features I love to have in my language,
such as iterators, exception handling, garbage collection.

Then, I reorganized the features of Perl into a class library,
and implemented them. I posted Ruby 0.95 to
the Japanese domestic newsgroups in Dec. 1995.
\end{quote}

\section{“Purely object-oriented”}
\label{sec:org7cbd489}
What does it mean when I have said, both about Ruby and Scala,
that they are purely object oriented?

Quite simply, that all \emph{data} is represented as an \emph{object,
and all operations are /methods}!

\subsection{Integers are objects, operations are methods}
\label{sec:orgdcfe76f}
For instance, consider integers, which in many languages are
a \emph{basic}, \emph{builtin} type that do not have an implementation
within the language. This is not the case in a pure language!
In both Ruby and Scala, integers are objects, and operations on them
are methods.

So code such as
\begin{minted}[breaklines=true]{ruby}
5 + 2
\end{minted}

\begin{minted}[breaklines=true]{scala}
5 + 2
\end{minted}

could instead be written
\begin{minted}[breaklines=true]{ruby}
5.+(2)
\end{minted}

\begin{minted}[breaklines=true]{scala}
5.+(2)
\end{minted}

that is, \texttt{+} is a method of the first argument, being passed
the second argument as a parameter.
The form \texttt{x ⊕ y} is just \emph{syntactic sugar}.

\subsection{Integers are objects, operations are \emph{messages}}
\label{sec:orgbfc6cce}
In Ruby, we can move one level of abstraction higher:
all data are objects, and all operations are \emph{messages between objects}.
This harkens back to SmallTalk, one of the founding languages
of the object oriented paradigm.
\begin{minted}[breaklines=true]{ruby}
5.send("==", 3)
\end{minted}

So even the form \texttt{x.⊕(y)} is syntactic sugar!

In Scala, moving to this message passing abstraction
is not possible, aat least not easily; why?
:TODO:

\section{Postfix forms}
\label{sec:org274bd90}
:TODO:

\section{Method naming conventions}
\label{sec:orga1f4c67}
By convention, methods ending with a \texttt{?} are predicates.
They do not \emph{necessarily} return a boolean, but should
return a “truth value” of some kind.
:TODO:

Methods ending with a \texttt{!} are \emph{descructive}; they modify the receiver.
:TODO:

Methods ending with a \texttt{=} indicate an \emph{assignment} method.
:TODO:

\section{Defining classes}
\label{sec:org0550c29}
\subsection{The basics}
\label{sec:org8ebb317}
A class declaration in Ruby for a class named \texttt{Name} is
begun by simply saying
\begin{verbatim}
class Name
\end{verbatim}

Instance variables
(whose value is unique per object of the class)
begin with a \texttt{@}.
We do not declare explicitely declare variables in Ruby,
but you can initialise them to “declare” them
(you don't need to though; they can be initialised inside
a constructor or other methods.)

Class variables
(whose value is shared by each object of the class)
begin with a \texttt{@@}.

It is common to want to define \emph{getters} and \emph{setters} for instance
variables in OO programming.
For example,
\begin{minted}[breaklines=true]{ruby}
class MyContainer
  
  def initialize(thing=nil) @thing = thing end

  def thing; @thing end

  # Assignment method syntax
  def thing=(thing) @thing = thing end
end

container = MyContainer.new()
container.thing = 5
puts(container.thing)
\end{minted}

Because this is so common, there is a shorthand
to avoid declaring these methods.
\begin{minted}[breaklines=true]{ruby}
class MyContainer
  attr_accessor :thing   # :thing is a symbol; essentially in interned string
  # attr_reader   provides only the getter
  # attr_writer   provides only the setter
\end{minted}

:TODO: initialize

\subsection{Inheritance}
\label{sec:org83d12ad}
:TODO:

\begin{minted}[breaklines=true]{ruby}
class C1

class C2 < C1
\end{minted}

\subsection{{\bfseries\sffamily TODO} Mixins}
\label{sec:orgb8ef083}
:TODO:
\end{document}
