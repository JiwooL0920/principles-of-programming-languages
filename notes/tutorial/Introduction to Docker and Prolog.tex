\documentclass[11pt]{article}

    \usepackage[breakable]{tcolorbox}
    \usepackage{parskip} % Stop auto-indenting (to mimic markdown behaviour)
    
    \usepackage{iftex}
    \ifPDFTeX
    	\usepackage[T1]{fontenc}
    	\usepackage{mathpazo}
    \else
    	\usepackage{fontspec}
    \fi

    % Basic figure setup, for now with no caption control since it's done
    % automatically by Pandoc (which extracts ![](path) syntax from Markdown).
    \usepackage{graphicx}
    % Maintain compatibility with old templates. Remove in nbconvert 6.0
    \let\Oldincludegraphics\includegraphics
    % Ensure that by default, figures have no caption (until we provide a
    % proper Figure object with a Caption API and a way to capture that
    % in the conversion process - todo).
    \usepackage{caption}
    \DeclareCaptionFormat{nocaption}{}
    \captionsetup{format=nocaption,aboveskip=0pt,belowskip=0pt}

    \usepackage{float}
    \floatplacement{figure}{H} % forces figures to be placed at the correct location
    \usepackage{xcolor} % Allow colors to be defined
    \usepackage{enumerate} % Needed for markdown enumerations to work
    \usepackage{geometry} % Used to adjust the document margins
    \usepackage{amsmath} % Equations
    \usepackage{amssymb} % Equations
    \usepackage{textcomp} % defines textquotesingle
    % Hack from http://tex.stackexchange.com/a/47451/13684:
    \AtBeginDocument{%
        \def\PYZsq{\textquotesingle}% Upright quotes in Pygmentized code
    }
    \usepackage{upquote} % Upright quotes for verbatim code
    \usepackage{eurosym} % defines \euro
    \usepackage[mathletters]{ucs} % Extended unicode (utf-8) support
    \usepackage{fancyvrb} % verbatim replacement that allows latex
    \usepackage{grffile} % extends the file name processing of package graphics 
                         % to support a larger range
    \makeatletter % fix for old versions of grffile with XeLaTeX
    \@ifpackagelater{grffile}{2019/11/01}
    {
      % Do nothing on new versions
    }
    {
      \def\Gread@@xetex#1{%
        \IfFileExists{"\Gin@base".bb}%
        {\Gread@eps{\Gin@base.bb}}%
        {\Gread@@xetex@aux#1}%
      }
    }
    \makeatother
    \usepackage[Export]{adjustbox} % Used to constrain images to a maximum size
    \adjustboxset{max size={0.9\linewidth}{0.9\paperheight}}

    % The hyperref package gives us a pdf with properly built
    % internal navigation ('pdf bookmarks' for the table of contents,
    % internal cross-reference links, web links for URLs, etc.)
    \usepackage{hyperref}
    % The default LaTeX title has an obnoxious amount of whitespace. By default,
    % titling removes some of it. It also provides customization options.
    \usepackage{titling}
    \usepackage{longtable} % longtable support required by pandoc >1.10
    \usepackage{booktabs}  % table support for pandoc > 1.12.2
    \usepackage[inline]{enumitem} % IRkernel/repr support (it uses the enumerate* environment)
    \usepackage[normalem]{ulem} % ulem is needed to support strikethroughs (\sout)
                                % normalem makes italics be italics, not underlines
    \usepackage{mathrsfs}
    

    
    % Colors for the hyperref package
    \definecolor{urlcolor}{rgb}{0,.145,.698}
    \definecolor{linkcolor}{rgb}{.71,0.21,0.01}
    \definecolor{citecolor}{rgb}{.12,.54,.11}

    % ANSI colors
    \definecolor{ansi-black}{HTML}{3E424D}
    \definecolor{ansi-black-intense}{HTML}{282C36}
    \definecolor{ansi-red}{HTML}{E75C58}
    \definecolor{ansi-red-intense}{HTML}{B22B31}
    \definecolor{ansi-green}{HTML}{00A250}
    \definecolor{ansi-green-intense}{HTML}{007427}
    \definecolor{ansi-yellow}{HTML}{DDB62B}
    \definecolor{ansi-yellow-intense}{HTML}{B27D12}
    \definecolor{ansi-blue}{HTML}{208FFB}
    \definecolor{ansi-blue-intense}{HTML}{0065CA}
    \definecolor{ansi-magenta}{HTML}{D160C4}
    \definecolor{ansi-magenta-intense}{HTML}{A03196}
    \definecolor{ansi-cyan}{HTML}{60C6C8}
    \definecolor{ansi-cyan-intense}{HTML}{258F8F}
    \definecolor{ansi-white}{HTML}{C5C1B4}
    \definecolor{ansi-white-intense}{HTML}{A1A6B2}
    \definecolor{ansi-default-inverse-fg}{HTML}{FFFFFF}
    \definecolor{ansi-default-inverse-bg}{HTML}{000000}

    % common color for the border for error outputs.
    \definecolor{outerrorbackground}{HTML}{FFDFDF}

    % commands and environments needed by pandoc snippets
    % extracted from the output of `pandoc -s`
    \providecommand{\tightlist}{%
      \setlength{\itemsep}{0pt}\setlength{\parskip}{0pt}}
    \DefineVerbatimEnvironment{Highlighting}{Verbatim}{commandchars=\\\{\}}
    % Add ',fontsize=\small' for more characters per line
    \newenvironment{Shaded}{}{}
    \newcommand{\KeywordTok}[1]{\textcolor[rgb]{0.00,0.44,0.13}{\textbf{{#1}}}}
    \newcommand{\DataTypeTok}[1]{\textcolor[rgb]{0.56,0.13,0.00}{{#1}}}
    \newcommand{\DecValTok}[1]{\textcolor[rgb]{0.25,0.63,0.44}{{#1}}}
    \newcommand{\BaseNTok}[1]{\textcolor[rgb]{0.25,0.63,0.44}{{#1}}}
    \newcommand{\FloatTok}[1]{\textcolor[rgb]{0.25,0.63,0.44}{{#1}}}
    \newcommand{\CharTok}[1]{\textcolor[rgb]{0.25,0.44,0.63}{{#1}}}
    \newcommand{\StringTok}[1]{\textcolor[rgb]{0.25,0.44,0.63}{{#1}}}
    \newcommand{\CommentTok}[1]{\textcolor[rgb]{0.38,0.63,0.69}{\textit{{#1}}}}
    \newcommand{\OtherTok}[1]{\textcolor[rgb]{0.00,0.44,0.13}{{#1}}}
    \newcommand{\AlertTok}[1]{\textcolor[rgb]{1.00,0.00,0.00}{\textbf{{#1}}}}
    \newcommand{\FunctionTok}[1]{\textcolor[rgb]{0.02,0.16,0.49}{{#1}}}
    \newcommand{\RegionMarkerTok}[1]{{#1}}
    \newcommand{\ErrorTok}[1]{\textcolor[rgb]{1.00,0.00,0.00}{\textbf{{#1}}}}
    \newcommand{\NormalTok}[1]{{#1}}
    
    % Additional commands for more recent versions of Pandoc
    \newcommand{\ConstantTok}[1]{\textcolor[rgb]{0.53,0.00,0.00}{{#1}}}
    \newcommand{\SpecialCharTok}[1]{\textcolor[rgb]{0.25,0.44,0.63}{{#1}}}
    \newcommand{\VerbatimStringTok}[1]{\textcolor[rgb]{0.25,0.44,0.63}{{#1}}}
    \newcommand{\SpecialStringTok}[1]{\textcolor[rgb]{0.73,0.40,0.53}{{#1}}}
    \newcommand{\ImportTok}[1]{{#1}}
    \newcommand{\DocumentationTok}[1]{\textcolor[rgb]{0.73,0.13,0.13}{\textit{{#1}}}}
    \newcommand{\AnnotationTok}[1]{\textcolor[rgb]{0.38,0.63,0.69}{\textbf{\textit{{#1}}}}}
    \newcommand{\CommentVarTok}[1]{\textcolor[rgb]{0.38,0.63,0.69}{\textbf{\textit{{#1}}}}}
    \newcommand{\VariableTok}[1]{\textcolor[rgb]{0.10,0.09,0.49}{{#1}}}
    \newcommand{\ControlFlowTok}[1]{\textcolor[rgb]{0.00,0.44,0.13}{\textbf{{#1}}}}
    \newcommand{\OperatorTok}[1]{\textcolor[rgb]{0.40,0.40,0.40}{{#1}}}
    \newcommand{\BuiltInTok}[1]{{#1}}
    \newcommand{\ExtensionTok}[1]{{#1}}
    \newcommand{\PreprocessorTok}[1]{\textcolor[rgb]{0.74,0.48,0.00}{{#1}}}
    \newcommand{\AttributeTok}[1]{\textcolor[rgb]{0.49,0.56,0.16}{{#1}}}
    \newcommand{\InformationTok}[1]{\textcolor[rgb]{0.38,0.63,0.69}{\textbf{\textit{{#1}}}}}
    \newcommand{\WarningTok}[1]{\textcolor[rgb]{0.38,0.63,0.69}{\textbf{\textit{{#1}}}}}
    
    
    % Define a nice break command that doesn't care if a line doesn't already
    % exist.
    \def\br{\hspace*{\fill} \\* }
    % Math Jax compatibility definitions
    \def\gt{>}
    \def\lt{<}
    \let\Oldtex\TeX
    \let\Oldlatex\LaTeX
    \renewcommand{\TeX}{\textrm{\Oldtex}}
    \renewcommand{\LaTeX}{\textrm{\Oldlatex}}
    % Document parameters
    % Document title
    \title{Introduction to Docker and Prolog}
    
    
    
    
    
% Pygments definitions
\makeatletter
\def\PY@reset{\let\PY@it=\relax \let\PY@bf=\relax%
    \let\PY@ul=\relax \let\PY@tc=\relax%
    \let\PY@bc=\relax \let\PY@ff=\relax}
\def\PY@tok#1{\csname PY@tok@#1\endcsname}
\def\PY@toks#1+{\ifx\relax#1\empty\else%
    \PY@tok{#1}\expandafter\PY@toks\fi}
\def\PY@do#1{\PY@bc{\PY@tc{\PY@ul{%
    \PY@it{\PY@bf{\PY@ff{#1}}}}}}}
\def\PY#1#2{\PY@reset\PY@toks#1+\relax+\PY@do{#2}}

\expandafter\def\csname PY@tok@w\endcsname{\def\PY@tc##1{\textcolor[rgb]{0.73,0.73,0.73}{##1}}}
\expandafter\def\csname PY@tok@c\endcsname{\let\PY@it=\textit\def\PY@tc##1{\textcolor[rgb]{0.25,0.50,0.50}{##1}}}
\expandafter\def\csname PY@tok@cp\endcsname{\def\PY@tc##1{\textcolor[rgb]{0.74,0.48,0.00}{##1}}}
\expandafter\def\csname PY@tok@k\endcsname{\let\PY@bf=\textbf\def\PY@tc##1{\textcolor[rgb]{0.00,0.50,0.00}{##1}}}
\expandafter\def\csname PY@tok@kp\endcsname{\def\PY@tc##1{\textcolor[rgb]{0.00,0.50,0.00}{##1}}}
\expandafter\def\csname PY@tok@kt\endcsname{\def\PY@tc##1{\textcolor[rgb]{0.69,0.00,0.25}{##1}}}
\expandafter\def\csname PY@tok@o\endcsname{\def\PY@tc##1{\textcolor[rgb]{0.40,0.40,0.40}{##1}}}
\expandafter\def\csname PY@tok@ow\endcsname{\let\PY@bf=\textbf\def\PY@tc##1{\textcolor[rgb]{0.67,0.13,1.00}{##1}}}
\expandafter\def\csname PY@tok@nb\endcsname{\def\PY@tc##1{\textcolor[rgb]{0.00,0.50,0.00}{##1}}}
\expandafter\def\csname PY@tok@nf\endcsname{\def\PY@tc##1{\textcolor[rgb]{0.00,0.00,1.00}{##1}}}
\expandafter\def\csname PY@tok@nc\endcsname{\let\PY@bf=\textbf\def\PY@tc##1{\textcolor[rgb]{0.00,0.00,1.00}{##1}}}
\expandafter\def\csname PY@tok@nn\endcsname{\let\PY@bf=\textbf\def\PY@tc##1{\textcolor[rgb]{0.00,0.00,1.00}{##1}}}
\expandafter\def\csname PY@tok@ne\endcsname{\let\PY@bf=\textbf\def\PY@tc##1{\textcolor[rgb]{0.82,0.25,0.23}{##1}}}
\expandafter\def\csname PY@tok@nv\endcsname{\def\PY@tc##1{\textcolor[rgb]{0.10,0.09,0.49}{##1}}}
\expandafter\def\csname PY@tok@no\endcsname{\def\PY@tc##1{\textcolor[rgb]{0.53,0.00,0.00}{##1}}}
\expandafter\def\csname PY@tok@nl\endcsname{\def\PY@tc##1{\textcolor[rgb]{0.63,0.63,0.00}{##1}}}
\expandafter\def\csname PY@tok@ni\endcsname{\let\PY@bf=\textbf\def\PY@tc##1{\textcolor[rgb]{0.60,0.60,0.60}{##1}}}
\expandafter\def\csname PY@tok@na\endcsname{\def\PY@tc##1{\textcolor[rgb]{0.49,0.56,0.16}{##1}}}
\expandafter\def\csname PY@tok@nt\endcsname{\let\PY@bf=\textbf\def\PY@tc##1{\textcolor[rgb]{0.00,0.50,0.00}{##1}}}
\expandafter\def\csname PY@tok@nd\endcsname{\def\PY@tc##1{\textcolor[rgb]{0.67,0.13,1.00}{##1}}}
\expandafter\def\csname PY@tok@s\endcsname{\def\PY@tc##1{\textcolor[rgb]{0.73,0.13,0.13}{##1}}}
\expandafter\def\csname PY@tok@sd\endcsname{\let\PY@it=\textit\def\PY@tc##1{\textcolor[rgb]{0.73,0.13,0.13}{##1}}}
\expandafter\def\csname PY@tok@si\endcsname{\let\PY@bf=\textbf\def\PY@tc##1{\textcolor[rgb]{0.73,0.40,0.53}{##1}}}
\expandafter\def\csname PY@tok@se\endcsname{\let\PY@bf=\textbf\def\PY@tc##1{\textcolor[rgb]{0.73,0.40,0.13}{##1}}}
\expandafter\def\csname PY@tok@sr\endcsname{\def\PY@tc##1{\textcolor[rgb]{0.73,0.40,0.53}{##1}}}
\expandafter\def\csname PY@tok@ss\endcsname{\def\PY@tc##1{\textcolor[rgb]{0.10,0.09,0.49}{##1}}}
\expandafter\def\csname PY@tok@sx\endcsname{\def\PY@tc##1{\textcolor[rgb]{0.00,0.50,0.00}{##1}}}
\expandafter\def\csname PY@tok@m\endcsname{\def\PY@tc##1{\textcolor[rgb]{0.40,0.40,0.40}{##1}}}
\expandafter\def\csname PY@tok@gh\endcsname{\let\PY@bf=\textbf\def\PY@tc##1{\textcolor[rgb]{0.00,0.00,0.50}{##1}}}
\expandafter\def\csname PY@tok@gu\endcsname{\let\PY@bf=\textbf\def\PY@tc##1{\textcolor[rgb]{0.50,0.00,0.50}{##1}}}
\expandafter\def\csname PY@tok@gd\endcsname{\def\PY@tc##1{\textcolor[rgb]{0.63,0.00,0.00}{##1}}}
\expandafter\def\csname PY@tok@gi\endcsname{\def\PY@tc##1{\textcolor[rgb]{0.00,0.63,0.00}{##1}}}
\expandafter\def\csname PY@tok@gr\endcsname{\def\PY@tc##1{\textcolor[rgb]{1.00,0.00,0.00}{##1}}}
\expandafter\def\csname PY@tok@ge\endcsname{\let\PY@it=\textit}
\expandafter\def\csname PY@tok@gs\endcsname{\let\PY@bf=\textbf}
\expandafter\def\csname PY@tok@gp\endcsname{\let\PY@bf=\textbf\def\PY@tc##1{\textcolor[rgb]{0.00,0.00,0.50}{##1}}}
\expandafter\def\csname PY@tok@go\endcsname{\def\PY@tc##1{\textcolor[rgb]{0.53,0.53,0.53}{##1}}}
\expandafter\def\csname PY@tok@gt\endcsname{\def\PY@tc##1{\textcolor[rgb]{0.00,0.27,0.87}{##1}}}
\expandafter\def\csname PY@tok@err\endcsname{\def\PY@bc##1{\setlength{\fboxsep}{0pt}\fcolorbox[rgb]{1.00,0.00,0.00}{1,1,1}{\strut ##1}}}
\expandafter\def\csname PY@tok@kc\endcsname{\let\PY@bf=\textbf\def\PY@tc##1{\textcolor[rgb]{0.00,0.50,0.00}{##1}}}
\expandafter\def\csname PY@tok@kd\endcsname{\let\PY@bf=\textbf\def\PY@tc##1{\textcolor[rgb]{0.00,0.50,0.00}{##1}}}
\expandafter\def\csname PY@tok@kn\endcsname{\let\PY@bf=\textbf\def\PY@tc##1{\textcolor[rgb]{0.00,0.50,0.00}{##1}}}
\expandafter\def\csname PY@tok@kr\endcsname{\let\PY@bf=\textbf\def\PY@tc##1{\textcolor[rgb]{0.00,0.50,0.00}{##1}}}
\expandafter\def\csname PY@tok@bp\endcsname{\def\PY@tc##1{\textcolor[rgb]{0.00,0.50,0.00}{##1}}}
\expandafter\def\csname PY@tok@fm\endcsname{\def\PY@tc##1{\textcolor[rgb]{0.00,0.00,1.00}{##1}}}
\expandafter\def\csname PY@tok@vc\endcsname{\def\PY@tc##1{\textcolor[rgb]{0.10,0.09,0.49}{##1}}}
\expandafter\def\csname PY@tok@vg\endcsname{\def\PY@tc##1{\textcolor[rgb]{0.10,0.09,0.49}{##1}}}
\expandafter\def\csname PY@tok@vi\endcsname{\def\PY@tc##1{\textcolor[rgb]{0.10,0.09,0.49}{##1}}}
\expandafter\def\csname PY@tok@vm\endcsname{\def\PY@tc##1{\textcolor[rgb]{0.10,0.09,0.49}{##1}}}
\expandafter\def\csname PY@tok@sa\endcsname{\def\PY@tc##1{\textcolor[rgb]{0.73,0.13,0.13}{##1}}}
\expandafter\def\csname PY@tok@sb\endcsname{\def\PY@tc##1{\textcolor[rgb]{0.73,0.13,0.13}{##1}}}
\expandafter\def\csname PY@tok@sc\endcsname{\def\PY@tc##1{\textcolor[rgb]{0.73,0.13,0.13}{##1}}}
\expandafter\def\csname PY@tok@dl\endcsname{\def\PY@tc##1{\textcolor[rgb]{0.73,0.13,0.13}{##1}}}
\expandafter\def\csname PY@tok@s2\endcsname{\def\PY@tc##1{\textcolor[rgb]{0.73,0.13,0.13}{##1}}}
\expandafter\def\csname PY@tok@sh\endcsname{\def\PY@tc##1{\textcolor[rgb]{0.73,0.13,0.13}{##1}}}
\expandafter\def\csname PY@tok@s1\endcsname{\def\PY@tc##1{\textcolor[rgb]{0.73,0.13,0.13}{##1}}}
\expandafter\def\csname PY@tok@mb\endcsname{\def\PY@tc##1{\textcolor[rgb]{0.40,0.40,0.40}{##1}}}
\expandafter\def\csname PY@tok@mf\endcsname{\def\PY@tc##1{\textcolor[rgb]{0.40,0.40,0.40}{##1}}}
\expandafter\def\csname PY@tok@mh\endcsname{\def\PY@tc##1{\textcolor[rgb]{0.40,0.40,0.40}{##1}}}
\expandafter\def\csname PY@tok@mi\endcsname{\def\PY@tc##1{\textcolor[rgb]{0.40,0.40,0.40}{##1}}}
\expandafter\def\csname PY@tok@il\endcsname{\def\PY@tc##1{\textcolor[rgb]{0.40,0.40,0.40}{##1}}}
\expandafter\def\csname PY@tok@mo\endcsname{\def\PY@tc##1{\textcolor[rgb]{0.40,0.40,0.40}{##1}}}
\expandafter\def\csname PY@tok@ch\endcsname{\let\PY@it=\textit\def\PY@tc##1{\textcolor[rgb]{0.25,0.50,0.50}{##1}}}
\expandafter\def\csname PY@tok@cm\endcsname{\let\PY@it=\textit\def\PY@tc##1{\textcolor[rgb]{0.25,0.50,0.50}{##1}}}
\expandafter\def\csname PY@tok@cpf\endcsname{\let\PY@it=\textit\def\PY@tc##1{\textcolor[rgb]{0.25,0.50,0.50}{##1}}}
\expandafter\def\csname PY@tok@c1\endcsname{\let\PY@it=\textit\def\PY@tc##1{\textcolor[rgb]{0.25,0.50,0.50}{##1}}}
\expandafter\def\csname PY@tok@cs\endcsname{\let\PY@it=\textit\def\PY@tc##1{\textcolor[rgb]{0.25,0.50,0.50}{##1}}}

\def\PYZbs{\char`\\}
\def\PYZus{\char`\_}
\def\PYZob{\char`\{}
\def\PYZcb{\char`\}}
\def\PYZca{\char`\^}
\def\PYZam{\char`\&}
\def\PYZlt{\char`\<}
\def\PYZgt{\char`\>}
\def\PYZsh{\char`\#}
\def\PYZpc{\char`\%}
\def\PYZdl{\char`\$}
\def\PYZhy{\char`\-}
\def\PYZsq{\char`\'}
\def\PYZdq{\char`\"}
\def\PYZti{\char`\~}
% for compatibility with earlier versions
\def\PYZat{@}
\def\PYZlb{[}
\def\PYZrb{]}
\makeatother


    % For linebreaks inside Verbatim environment from package fancyvrb. 
    \makeatletter
        \newbox\Wrappedcontinuationbox 
        \newbox\Wrappedvisiblespacebox 
        \newcommand*\Wrappedvisiblespace {\textcolor{red}{\textvisiblespace}} 
        \newcommand*\Wrappedcontinuationsymbol {\textcolor{red}{\llap{\tiny$\m@th\hookrightarrow$}}} 
        \newcommand*\Wrappedcontinuationindent {3ex } 
        \newcommand*\Wrappedafterbreak {\kern\Wrappedcontinuationindent\copy\Wrappedcontinuationbox} 
        % Take advantage of the already applied Pygments mark-up to insert 
        % potential linebreaks for TeX processing. 
        %        {, <, #, %, $, ' and ": go to next line. 
        %        _, }, ^, &, >, - and ~: stay at end of broken line. 
        % Use of \textquotesingle for straight quote. 
        \newcommand*\Wrappedbreaksatspecials {% 
            \def\PYGZus{\discretionary{\char`\_}{\Wrappedafterbreak}{\char`\_}}% 
            \def\PYGZob{\discretionary{}{\Wrappedafterbreak\char`\{}{\char`\{}}% 
            \def\PYGZcb{\discretionary{\char`\}}{\Wrappedafterbreak}{\char`\}}}% 
            \def\PYGZca{\discretionary{\char`\^}{\Wrappedafterbreak}{\char`\^}}% 
            \def\PYGZam{\discretionary{\char`\&}{\Wrappedafterbreak}{\char`\&}}% 
            \def\PYGZlt{\discretionary{}{\Wrappedafterbreak\char`\<}{\char`\<}}% 
            \def\PYGZgt{\discretionary{\char`\>}{\Wrappedafterbreak}{\char`\>}}% 
            \def\PYGZsh{\discretionary{}{\Wrappedafterbreak\char`\#}{\char`\#}}% 
            \def\PYGZpc{\discretionary{}{\Wrappedafterbreak\char`\%}{\char`\%}}% 
            \def\PYGZdl{\discretionary{}{\Wrappedafterbreak\char`\$}{\char`\$}}% 
            \def\PYGZhy{\discretionary{\char`\-}{\Wrappedafterbreak}{\char`\-}}% 
            \def\PYGZsq{\discretionary{}{\Wrappedafterbreak\textquotesingle}{\textquotesingle}}% 
            \def\PYGZdq{\discretionary{}{\Wrappedafterbreak\char`\"}{\char`\"}}% 
            \def\PYGZti{\discretionary{\char`\~}{\Wrappedafterbreak}{\char`\~}}% 
        } 
        % Some characters . , ; ? ! / are not pygmentized. 
        % This macro makes them "active" and they will insert potential linebreaks 
        \newcommand*\Wrappedbreaksatpunct {% 
            \lccode`\~`\.\lowercase{\def~}{\discretionary{\hbox{\char`\.}}{\Wrappedafterbreak}{\hbox{\char`\.}}}% 
            \lccode`\~`\,\lowercase{\def~}{\discretionary{\hbox{\char`\,}}{\Wrappedafterbreak}{\hbox{\char`\,}}}% 
            \lccode`\~`\;\lowercase{\def~}{\discretionary{\hbox{\char`\;}}{\Wrappedafterbreak}{\hbox{\char`\;}}}% 
            \lccode`\~`\:\lowercase{\def~}{\discretionary{\hbox{\char`\:}}{\Wrappedafterbreak}{\hbox{\char`\:}}}% 
            \lccode`\~`\?\lowercase{\def~}{\discretionary{\hbox{\char`\?}}{\Wrappedafterbreak}{\hbox{\char`\?}}}% 
            \lccode`\~`\!\lowercase{\def~}{\discretionary{\hbox{\char`\!}}{\Wrappedafterbreak}{\hbox{\char`\!}}}% 
            \lccode`\~`\/\lowercase{\def~}{\discretionary{\hbox{\char`\/}}{\Wrappedafterbreak}{\hbox{\char`\/}}}% 
            \catcode`\.\active
            \catcode`\,\active 
            \catcode`\;\active
            \catcode`\:\active
            \catcode`\?\active
            \catcode`\!\active
            \catcode`\/\active 
            \lccode`\~`\~ 	
        }
    \makeatother

    \let\OriginalVerbatim=\Verbatim
    \makeatletter
    \renewcommand{\Verbatim}[1][1]{%
        %\parskip\z@skip
        \sbox\Wrappedcontinuationbox {\Wrappedcontinuationsymbol}%
        \sbox\Wrappedvisiblespacebox {\FV@SetupFont\Wrappedvisiblespace}%
        \def\FancyVerbFormatLine ##1{\hsize\linewidth
            \vtop{\raggedright\hyphenpenalty\z@\exhyphenpenalty\z@
                \doublehyphendemerits\z@\finalhyphendemerits\z@
                \strut ##1\strut}%
        }%
        % If the linebreak is at a space, the latter will be displayed as visible
        % space at end of first line, and a continuation symbol starts next line.
        % Stretch/shrink are however usually zero for typewriter font.
        \def\FV@Space {%
            \nobreak\hskip\z@ plus\fontdimen3\font minus\fontdimen4\font
            \discretionary{\copy\Wrappedvisiblespacebox}{\Wrappedafterbreak}
            {\kern\fontdimen2\font}%
        }%
        
        % Allow breaks at special characters using \PYG... macros.
        \Wrappedbreaksatspecials
        % Breaks at punctuation characters . , ; ? ! and / need catcode=\active 	
        \OriginalVerbatim[#1,codes*=\Wrappedbreaksatpunct]%
    }
    \makeatother

    % Exact colors from NB
    \definecolor{incolor}{HTML}{303F9F}
    \definecolor{outcolor}{HTML}{D84315}
    \definecolor{cellborder}{HTML}{CFCFCF}
    \definecolor{cellbackground}{HTML}{F7F7F7}
    
    % prompt
    \makeatletter
    \newcommand{\boxspacing}{\kern\kvtcb@left@rule\kern\kvtcb@boxsep}
    \makeatother
    \newcommand{\prompt}[4]{
        {\ttfamily\llap{{\color{#2}[#3]:\hspace{3pt}#4}}\vspace{-\baselineskip}}
    }
    

    
    % Prevent overflowing lines due to hard-to-break entities
    \sloppy 
    % Setup hyperref package
    \hypersetup{
      breaklinks=true,  % so long urls are correctly broken across lines
      colorlinks=true,
      urlcolor=urlcolor,
      linkcolor=linkcolor,
      citecolor=citecolor,
      }
    % Slightly bigger margins than the latex defaults
    
    \geometry{verbose,tmargin=1in,bmargin=1in,lmargin=1in,rmargin=1in}
    
    

\begin{document}
    
    \maketitle
    
    

    
    \hypertarget{introduction-to-docker-and-how-to-use-it.}{%
\section{Introduction to Docker and how to use
it.}\label{introduction-to-docker-and-how-to-use-it.}}

Docker is a set of platform as a service (PaaS) products that use
OS-level virtualization to deliver software in packages called
containers.Containers are isolated from one another and bundle their own
software, libraries and configuration files; they can communicate with
each other through well-defined channels. All containers are run by a
single operating system kernel and therefore use fewer resources than
virtual machines.

\href{https://www.docker.com/}{Docker is free to download}.

\hypertarget{install-docker}{%
\subsection{Install Docker}\label{install-docker}}

\hypertarget{how-to-install-docker-in-windows.}{%
\subsubsection{How to install docker in
windows.}\label{how-to-install-docker-in-windows.}}

For this purpose it is better to install Ubuntu as a subsystem for
windows. Then we can run docker using the terminal of the Subsystem we
installed. Lets have a look and see how we can do this.

\hypertarget{how-to-install-docker-in-mac.}{%
\subsubsection{How to install docker in
Mac.}\label{how-to-install-docker-in-mac.}}

Installing docker desktop in Mac is straightforward. Lets have look at
this process as well.

\hypertarget{how-to-install-docker-in-ubuntu}{%
\subsubsection{How to install docker in
ubuntu}\label{how-to-install-docker-in-ubuntu}}

\href{https://docs.docker.com/engine/install/ubuntu/}{Installing docker
on Ubuntu}

\hypertarget{interact-with-docker}{%
\subsection{Interact with Docker}\label{interact-with-docker}}

In order to use Docker we, need to first generate images with an OS
kernel and configuration that necessary for an specific operation. We
can do this by defining the \texttt{Dockerfile} Lets have a close look
at an example to understand how it works:

    \begin{Shaded}
\begin{Highlighting}[]

\CommentTok{\# This Dockerfile has two required ARGs to determine which base image}
\CommentTok{\# to use for the JDK and which sbt version to install.}

\CommentTok{\# Define the argument for openjdk version}
\KeywordTok{ARG}\NormalTok{ OPENJDK\_TAG=8u232}
\CommentTok{\# Do the packaging based on openjdk}
\KeywordTok{FROM}\NormalTok{ openjdk:8u232}

\CommentTok{\# Set the name of the maintainers}
\KeywordTok{MAINTAINER}\NormalTok{ Habib Ghaffari Hadigheh, Mark Armstrong \textless{}ghaffh1@mcmaster.ca, armstmp@mcmaster.ca\textgreater{}}

\KeywordTok{RUN}\NormalTok{ apt{-}get update \&\& \textbackslash{}}
\NormalTok{  apt{-}get install scala {-}y \&\& \textbackslash{}}
\NormalTok{  apt{-}get install {-}y curl \&\& \textbackslash{}}
\NormalTok{  sh {-}c }\StringTok{\textquotesingle{}(echo "\#!/usr/bin/env sh" \&\& \textbackslash{}}
\StringTok{  curl {-}L https://github.com/lihaoyi/Ammonite/releases/download/2.1.1/2.12{-}2.1.1) \textgreater{} /usr/local/bin/amm \&\& \textbackslash{}}
\StringTok{  chmod +x /usr/local/bin/amm\textquotesingle{}}

\CommentTok{\# Set the working directory}
\KeywordTok{WORKDIR}\NormalTok{ /opt/h1}
\end{Highlighting}
\end{Shaded}

    \hypertarget{basic-commands-to-work-with-docker}{%
\subsection{Basic commands to work with
Docker}\label{basic-commands-to-work-with-docker}}

If you want to use docker as tool for implementation, you may need to
know lots of commands, here is
\href{https://www.docker.com/sites/default/files/d8/2019-09/docker-cheat-sheet.pdf}{cheat
sheet} for this purpose. But in this course it is not necessery to learn
and remeber many complicated processes. Here is a short list of command
you may/may not use in this course:

\begin{itemize}
\tightlist
\item
  Build
\end{itemize}

\begin{verbatim}
docker build [OPTIONS] PATH | URL | -
\end{verbatim}

\begin{itemize}
\tightlist
\item
  Run
\end{itemize}

\begin{verbatim}
docker run [OPTIONS] IMAGE [COMMAND] [ARG...]
\end{verbatim}

\begin{itemize}
\tightlist
\item
  See the list of images:
\end{itemize}

\begin{verbatim}
docker images
\end{verbatim}

\begin{itemize}
\tightlist
\item
  See the list of containers
\end{itemize}

\begin{verbatim}
docker ps [OPTIONS]
\end{verbatim}

Learning all the commands and optoins take time. The best resource for
biggners to start with docker is
\href{https://kodekloud.com/p/docker-for-the-absolute-beginner-hands-on}{this}
online free course.

\hypertarget{how-to-use-it-for-the-purpose-of-this-course}{%
\subsection{How to use it for the purpose of this
course}\label{how-to-use-it-for-the-purpose-of-this-course}}

We already defined the \texttt{Dockerfile} with all necessary packages
and library necassery for you. We also definde a
\texttt{docker-compose.yml} file that will help you in the process of
building the images, as well as running/stoping the containers.

Lets see how you can do that.

\href{https://armkeh.github.io/principles-of-programming-languages/homework/h1.html\#Testing}{Here}
is the link of the Docker test for h1.

Now we are going to build and run the test to see how it works.

\href{https://github.com/armkeh/principles-of-programming-languages/tree/master/homework/testing/h1}{h1
test using docker}

    \hypertarget{learning-inference-rules}{%
\section{Learning Inference Rules}\label{learning-inference-rules}}

We are using inference rules in this course for two main reasons:

\begin{enumerate}
\def\labelenumi{\arabic{enumi}.}
\tightlist
\item
  To understand the semantics of programming languages.
\item
  Discuss the type systems.
\end{enumerate}

\begin{itemize}
\tightlist
\item
  Prolog is a programming language that is all about inference rules.
  That is why we are covering it at the begging of the course this year.
\end{itemize}

\hypertarget{prolog}{%
\subsection{Prolog}\label{prolog}}

Prolog programs are simply databases of inference rules, also called
clauses. Let have a look at a simple inference rule.

\(\cfrac{A_1\text{ }A_2\text{ }...\text{ }A_n}{B}\)

Lets see how we can write it down in Prolog:

    \begin{Shaded}
\begin{Highlighting}[]
\NormalTok{b }\KeywordTok{:{-}}\NormalTok{ a1}\KeywordTok{,}\NormalTok{ a2}\KeywordTok{,}\NormalTok{ a3}\KeywordTok{,}\NormalTok{ a4}\KeywordTok{,}\NormalTok{ a5}\KeywordTok{.}
\end{Highlighting}
\end{Shaded}

    We can seperate the permisis by \texttt{,}. This rule states that \(b\)
is true if all of the \(a_i\) are true. So we can think of \texttt{:-}
as \(\leftarrow\). Notice the \textbf{period} ending the rule.

Now lets have a look at a very simple axiom:

\(\begin{array}{c}  \\  \hline  C  \end{array}\)

We can write this in prolog as

    \begin{Shaded}
\begin{Highlighting}[]
\NormalTok{c }\KeywordTok{:{-}} \KeywordTok{true.}
\end{Highlighting}
\end{Shaded}

or more simply,

\begin{Shaded}
\begin{Highlighting}[]
\NormalTok{c}\KeywordTok{.}
\end{Highlighting}
\end{Shaded}

    How we can get output from sequence of inference rules?

\hypertarget{get-output-from-sequence-of-inference-rules}{%
\subsection{Get output from sequence of inference
rules}\label{get-output-from-sequence-of-inference-rules}}

To interact with prolog programs, we make queries, to which Prolog
reponds by checking its inference rule database to determine possible
anwsers. Lets have a look at one example:

\hypertarget{list-catenatoin}{%
\subsubsection{List catenatoin}\label{list-catenatoin}}

List catenation in SWI prolog is a \textbf{ternary predicate}. Here is
how it's written

\begin{Shaded}
\begin{Highlighting}[]
\NormalTok{append(}\DataTypeTok{X}\KeywordTok{,}\DataTypeTok{Y}\KeywordTok{,}\DataTypeTok{Z}\NormalTok{)}
\end{Highlighting}
\end{Shaded}

the rule of which enforce that \texttt{Z} is the result of catenating
\texttt{X} and \texttt{Y}.

lets have a look and see which kinds of queries we cand make:

\begin{itemize}
\tightlist
\item
  Is \texttt{{[}1,2,3,4{]}} the result of catenating \texttt{{[}1,2{]}},
  and \texttt{{[}3,4{]}}?
\end{itemize}

    \begin{tcolorbox}[breakable, size=fbox, boxrule=1pt, pad at break*=1mm,colback=cellbackground, colframe=cellborder]
\prompt{In}{incolor}{1}{\boxspacing}
\begin{Verbatim}[commandchars=\\\{\}]
\PY{o}{\PYZpc{}\PYZpc{}}\PY{k}{script} swipl \PYZhy{}q
append([1,2],[3,4],[1,2,3,4]).
\end{Verbatim}
\end{tcolorbox}

    \begin{Verbatim}[commandchars=\\\{\}]
true.


    \end{Verbatim}

    \begin{itemize}
\tightlist
\item
  What are the possible values of \texttt{Z} for which \texttt{Z} is the
  catenation of \texttt{{[}1,2,3{]}} and \texttt{{[}4,5,6{]}}?
\end{itemize}

    \begin{tcolorbox}[breakable, size=fbox, boxrule=1pt, pad at break*=1mm,colback=cellbackground, colframe=cellborder]
\prompt{In}{incolor}{2}{\boxspacing}
\begin{Verbatim}[commandchars=\\\{\}]
\PY{o}{\PYZpc{}\PYZpc{}}\PY{k}{script} swipl \PYZhy{}q
append([1,2,3],[4,5,6],Z).
\end{Verbatim}
\end{tcolorbox}

    \begin{Verbatim}[commandchars=\\\{\}]
Z = [1, 2, 3, 4, 5, 6].


    \end{Verbatim}

    What are the possible values of \texttt{X} and \texttt{Y} so that, when
they are catenated, the result is \texttt{{[}1,2,3,4,5,6{]}}?

    \begin{tcolorbox}[breakable, size=fbox, boxrule=1pt, pad at break*=1mm,colback=cellbackground, colframe=cellborder]
\prompt{In}{incolor}{3}{\boxspacing}
\begin{Verbatim}[commandchars=\\\{\}]
\PY{o}{\PYZpc{}\PYZpc{}}\PY{k}{script} swipl \PYZhy{}q
append(X,Y, [1,2,3,4,5,6]).
\end{Verbatim}
\end{tcolorbox}

    \begin{Verbatim}[commandchars=\\\{\}]
X = [],
Y = [1, 2, 3, 4, 5, 6]
    \end{Verbatim}

    \begin{Verbatim}[commandchars=\\\{\}]

ERROR: Type error: `character\_code' expected, found `-1' (an integer)
ERROR: In:
ERROR:   [11] char\_code(\_23930,-1)
ERROR:   [10] '\$in\_reply'(-1,'?h') at /usr/local/Cellar/swi-
prolog/8.2.1/libexec/lib/swipl/boot/init.pl:911
    \end{Verbatim}

    The reponse will be:

\begin{Shaded}
\begin{Highlighting}[]

\DataTypeTok{X} \KeywordTok{=}\NormalTok{ []}\KeywordTok{,}
\DataTypeTok{Y} \KeywordTok{=}\NormalTok{ [}\DecValTok{1}\NormalTok{, }\DecValTok{2}\NormalTok{, }\DecValTok{3}\NormalTok{, }\DecValTok{4}\NormalTok{, }\DecValTok{5}\NormalTok{, }\DecValTok{6}\NormalTok{] }\KeywordTok{;}
\DataTypeTok{X} \KeywordTok{=}\NormalTok{ [}\DecValTok{1}\NormalTok{]}\KeywordTok{,}
\DataTypeTok{Y} \KeywordTok{=}\NormalTok{ [}\DecValTok{2}\NormalTok{, }\DecValTok{3}\NormalTok{, }\DecValTok{4}\NormalTok{, }\DecValTok{5}\NormalTok{, }\DecValTok{6}\NormalTok{] }\KeywordTok{;}
\DataTypeTok{X} \KeywordTok{=}\NormalTok{ [}\DecValTok{1}\NormalTok{, }\DecValTok{2}\NormalTok{]}\KeywordTok{,}
\DataTypeTok{Y} \KeywordTok{=}\NormalTok{ [}\DecValTok{3}\NormalTok{, }\DecValTok{4}\NormalTok{, }\DecValTok{5}\NormalTok{, }\DecValTok{6}\NormalTok{] }\KeywordTok{;}
\DataTypeTok{X} \KeywordTok{=}\NormalTok{ [}\DecValTok{1}\NormalTok{, }\DecValTok{2}\NormalTok{, }\DecValTok{3}\NormalTok{]}\KeywordTok{,}
\DataTypeTok{Y} \KeywordTok{=}\NormalTok{ [}\DecValTok{4}\NormalTok{, }\DecValTok{5}\NormalTok{, }\DecValTok{6}\NormalTok{] }\KeywordTok{;}
\DataTypeTok{X} \KeywordTok{=}\NormalTok{ [}\DecValTok{1}\NormalTok{, }\DecValTok{2}\NormalTok{, }\DecValTok{3}\NormalTok{, }\DecValTok{4}\NormalTok{]}\KeywordTok{,}
\DataTypeTok{Y} \KeywordTok{=}\NormalTok{ [}\DecValTok{5}\NormalTok{, }\DecValTok{6}\NormalTok{] }\KeywordTok{;}
\DataTypeTok{X} \KeywordTok{=}\NormalTok{ [}\DecValTok{1}\NormalTok{, }\DecValTok{2}\NormalTok{, }\DecValTok{3}\NormalTok{, }\DecValTok{4}\NormalTok{, }\DecValTok{5}\NormalTok{]}\KeywordTok{,}
\DataTypeTok{Y} \KeywordTok{=}\NormalTok{ [}\DecValTok{6}\NormalTok{] }\KeywordTok{;}
\DataTypeTok{X} \KeywordTok{=}\NormalTok{ [}\DecValTok{1}\NormalTok{, }\DecValTok{2}\NormalTok{, }\DecValTok{3}\NormalTok{, }\DecValTok{4}\NormalTok{, }\DecValTok{5}\NormalTok{, }\DecValTok{6}\NormalTok{]}\KeywordTok{,}
\DataTypeTok{Y} \KeywordTok{=}\NormalTok{ [] }\KeywordTok{;}
\end{Highlighting}
\end{Shaded}

    \textbf{Note:} In this way, we get several ``functions'' from one
predicate, depending upon what question(s) we ask!

    \hypertarget{names-kinds-of-tems}{%
\subsection{Names, kinds of tems}\label{names-kinds-of-tems}}

\begin{itemize}
\item
  In Prolog, any \textbf{name} begining with an upper case letter
  denotes a variable.
\item
  Names which begin with lower case letters are \textbf{atoms}. which
  are a type of constant value. Atoms may be used as the name of
  predicates.
\end{itemize}

\textbf{Note:} You should avoid calling atome functions, because they
are relations. Functions are special kind of relations/predicates that
are single-value.

\begin{itemize}
\tightlist
\item
  Character strings surrounded by single quotes, are also atoms. So we
  can write
\end{itemize}

\begin{Shaded}
\begin{Highlighting}[]
\StringTok{\textquotesingle{}}\ErrorTok{is}\AlertTok{ }\ErrorTok{an}\AlertTok{ }\ErrorTok{empty}\AlertTok{ }\ErrorTok{list}\StringTok{\textquotesingle{}}\NormalTok{([])}\KeywordTok{.}
\end{Highlighting}
\end{Shaded}

\begin{itemize}
\item
  As you would expect, Prolog also has numerical constants, such as
  \texttt{1} or \texttt{3.14}.
\item
  Aside from what described above, the remianing kind or Prolog term is
  a \emph{structure}, which has the fome below:
\end{itemize}

\begin{verbatim}
atom(term1,...,terml)
\end{verbatim}

As you can see the syntax is simple, the main important and probably
problmatic part of Prolog is simantics of it.

\hypertarget{interacting-with-prolog}{%
\subsection{Interacting with Prolog}\label{interacting-with-prolog}}

As we've said, a Prolog program consist of clauses (inference rules.) As
an example have a look at this clause:

\begin{Shaded}
\begin{Highlighting}[]

\NormalTok{head(}\DataTypeTok{X}\NormalTok{) }\KeywordTok{:{-}}\NormalTok{ body(}\DataTypeTok{X}\KeywordTok{,}\DataTypeTok{Y}\NormalTok{)}
\end{Highlighting}
\end{Shaded}

Which can be interpreted as below:

\(\forall X, head(X) \leftarrow (\exists Y, body(X,Y))\)

Then during computation, given this cluase and the goal
\texttt{head(X)}, the Prolog runtime is tasked with finding a
substitution for \texttt{Y} which makes \texttt{body(X,Y)} true.

We provide Prolog with goals through queries, usually by loading our
programs into the interactive query REPL, either by running

\begin{verbatim}
swipl my_program.pl
\end{verbatim}

from the command line, or

\begin{Shaded}
\begin{Highlighting}[]

\FunctionTok{?{-}}\NormalTok{ consult(}\StringTok{\textquotesingle{}}\ErrorTok{my\_program}\StringTok{.}\ErrorTok{pl}\StringTok{\textquotesingle{}}\NormalTok{)}
\end{Highlighting}
\end{Shaded}

once runing SWI Prolog.

We can also \texttt{assert} or \texttt{retract} rules in the query REPL.
if needed. Let have look at that.

    \begin{tcolorbox}[breakable, size=fbox, boxrule=1pt, pad at break*=1mm,colback=cellbackground, colframe=cellborder]
\prompt{In}{incolor}{4}{\boxspacing}
\begin{Verbatim}[commandchars=\\\{\}]
\PY{o}{\PYZpc{}\PYZpc{}}\PY{k}{script} swipl \PYZhy{}q
assert(c).
c.
retract(c).
c.
assert(d).
assert(c :\PYZhy{} d).
c.
\end{Verbatim}
\end{tcolorbox}

    \begin{Verbatim}[commandchars=\\\{\}]
true.

true.

true.

false.

true.

true.

true.


    \end{Verbatim}

    Also, use \texttt{listing} or \texttt{listing(name)} to see all given
clauses or clauses about the \texttt{name} predicate.

\hypertarget{unification}{%
\subsection{Unification}\label{unification}}

The computation of model of Prolog involves \emph{unification} of terms.
Terms unify if either: 1. They are equal, or 1. They contain variables
that can be \textbf{instantiated} in a way that makes the terms equal.

We saw an example of this when we are using
\texttt{append(X,Y,{[}1,2,3,4{]}}). Prolog tries to find us a possible
bining. If it cannot then it reply false.

So in general, unification involves searching for possible variable
binings, by making use of the clauses, and modus ponens

\((P \land P\Rightarrow Q) \Rightarrow Q\)

\hypertarget{the-goal-list}{%
\subsubsection{The goal list}\label{the-goal-list}}

Through this process, the single goal presented by a quey will usually
turn into a collection of goals; For instance if we query
\texttt{?-\ P(5).} and the search uses a clause

\begin{Shaded}
\begin{Highlighting}[]
\NormalTok{p(}\DataTypeTok{X}\NormalTok{) }\KeywordTok{:{-}} \DataTypeTok{Q}\NormalTok{(}\DataTypeTok{X}\NormalTok{)}\KeywordTok{,} \DataTypeTok{R}\NormalTok{(}\DataTypeTok{X}\NormalTok{)}\KeywordTok{.}
\end{Highlighting}
\end{Shaded}

then we now have as goals \texttt{q(X)} and \texttt{r(X)}.

Now the goals will change to \texttt{q(5)} and \texttt{r(5)}.

\hypertarget{backtracking}{%
\subsubsection{Backtracking}\label{backtracking}}

Now we are defining a small program:

$ a \leftarrow b \land c $

$ a \leftarrow b$

$ b = \top$

$ c = \bot$

Here is how we write it in prolog:

\begin{Shaded}
\begin{Highlighting}[]
\NormalTok{a }\KeywordTok{:{-}}\NormalTok{ b}\KeywordTok{,}\NormalTok{ c}\KeywordTok{.}
\NormalTok{a }\KeywordTok{:{-}}\NormalTok{ b}\KeywordTok{.}

\NormalTok{b}\KeywordTok{.}
\NormalTok{c }\KeywordTok{:{-}} \KeywordTok{false.}
\end{Highlighting}
\end{Shaded}

This will be the result of \texttt{?-\ trace.} command:

\begin{verbatim}
[trace]  ?- a.
   Call: (10) a ? creep
   Call: (11) b ? creep
   Exit: (11) b ? creep
   Call: (11) c ? creep
   Call: (12) false ? creep
   Fail: (12) false ? creep
   Fail: (11) c ? creep
   Redo: (10) a ? creep
   Call: (11) b ? creep
   Exit: (11) b ? creep
   Exit: (10) a ? creep
true.

\end{verbatim}

As Prolog runtime tries to prove a, it will use the first rule, and
\texttt{fail} (because in trying to prove \texttt{c}, it reaches
\texttt{false}, which it cannot prove.). At that point it has to
\textbf{backtrack}, and try a different clause to prove \texttt{a}.

In general runtime will backtrac serveral times during a proof, and it
keeps track of which clauses have been tried.

\hypertarget{swi-prologs-search-strategy}{%
\subsubsection{SWI Prolog's search
strategy}\label{swi-prologs-search-strategy}}

\begin{enumerate}
\def\labelenumi{\arabic{enumi}.}
\tightlist
\item
  Attempt to apply clauses in order from top to bottom (as in the source
  code).
\end{enumerate}

\begin{itemize}
\tightlist
\item
  Only backtrack and try other clauses after success and failure.
\end{itemize}

\begin{enumerate}
\def\labelenumi{\arabic{enumi}.}
\tightlist
\item
  Perform a depth first search to prove each goal.
\end{enumerate}

\begin{itemize}
\tightlist
\item
  So if the current goal are \texttt{b} and \texttt{c}, try to prove
  \texttt{b} before considering \texttt{c}.
\end{itemize}

\hypertarget{examining-the-search-strategy}{%
\subsubsection{Examining the search
strategy}\label{examining-the-search-strategy}}

In order to interactively see the process Prologe is using during
unification, use the trace. command in the REPL. Then each query will
result in a log of calls made and failures encoutered.

\hypertarget{equality}{%
\subsection{Equality}\label{equality}}

Prolog has an equality comparision, written simply \texttt{=} (not
\texttt{==}). \textbf{However}, this equality does not simplification.
So. for instance, if a variable \texttt{X} has been unified to value
\texttt{5}

\begin{Shaded}
\begin{Highlighting}[]
\DataTypeTok{X} \KeywordTok{=} \DecValTok{5}
\end{Highlighting}
\end{Shaded}

would be \texttt{true}. but

\begin{Shaded}
\begin{Highlighting}[]
\DataTypeTok{X}\KeywordTok{=} \DecValTok{2} \FunctionTok{+} \DecValTok{3}
\end{Highlighting}
\end{Shaded}

would be \texttt{false}.

Lets try it out very quickly:

    \begin{tcolorbox}[breakable, size=fbox, boxrule=1pt, pad at break*=1mm,colback=cellbackground, colframe=cellborder]
\prompt{In}{incolor}{5}{\boxspacing}
\begin{Verbatim}[commandchars=\\\{\}]
\PY{o}{\PYZpc{}\PYZpc{}}\PY{k}{script} swipl \PYZhy{}q
5 = 2 + 3.
\end{Verbatim}
\end{tcolorbox}

    \begin{Verbatim}[commandchars=\\\{\}]
false.


    \end{Verbatim}

    How we can do this comparison?

    This non-simplifying equality allows us to conisder construction of
terms, rather that just their value. But in case we want to actually
carry out arithmetic or calculate other values, we can use the
\texttt{is} comparison.

    \begin{tcolorbox}[breakable, size=fbox, boxrule=1pt, pad at break*=1mm,colback=cellbackground, colframe=cellborder]
\prompt{In}{incolor}{6}{\boxspacing}
\begin{Verbatim}[commandchars=\\\{\}]
\PY{o}{\PYZpc{}\PYZpc{}}\PY{k}{script} swipl \PYZhy{}q
5 is 2 + 3.
\end{Verbatim}
\end{tcolorbox}

    \begin{Verbatim}[commandchars=\\\{\}]
true.


    \end{Verbatim}

    \hypertarget{exerting-control-over-the-search-the-cut-operator}{%
\subsection{Exerting control over the search; the cut
operator}\label{exerting-control-over-the-search-the-cut-operator}}

In part because Prolog's searching mechanism can be naive, the
programmer is given a certain amount of control over the search.

The most important mechanism for controlling the search that we will see
is the cut.

A cut is written \texttt{!}, and can be understood as "no backtracking
is allowed to go beyond this point

\hypertarget{checking-for-divisors-of-a-number}{%
\subsection{Checking for divisors of a
number}\label{checking-for-divisors-of-a-number}}

\begin{Shaded}
\begin{Highlighting}[]
\NormalTok{hasDivisorLessThanOrEqualTo(}\DataTypeTok{\_}\KeywordTok{,}\DecValTok{1}\NormalTok{) }\KeywordTok{:{-}} \KeywordTok{!,} \KeywordTok{false.}
\NormalTok{hasDivisorLessThanOrEqualTo(}\DataTypeTok{X}\KeywordTok{,}\DataTypeTok{Y}\NormalTok{) }\KeywordTok{:{-}} \DecValTok{0} \DataTypeTok{is} \DataTypeTok{X} \DecValTok{mod} \DataTypeTok{Y}\KeywordTok{,} \KeywordTok{!.}
\NormalTok{hasDivisorLessThanOrEqualTo(}\DataTypeTok{X}\KeywordTok{,}\DataTypeTok{Y}\NormalTok{) }\KeywordTok{:{-}} \DataTypeTok{Z} \DataTypeTok{is} \DataTypeTok{Y} \DataTypeTok{{-}} \DecValTok{1}\KeywordTok{,}\NormalTok{ hasDivisorLessThanOrEqualTo(}\DataTypeTok{X}\KeywordTok{,}\DataTypeTok{Z}\NormalTok{)}\KeywordTok{.}
\end{Highlighting}
\end{Shaded}

Lets try this out to see what is the results:

\begin{verbatim}
\end{verbatim}


    % Add a bibliography block to the postdoc
    
    
    
\end{document}
