% Created 2020-09-29 Tue 23:17
% Intended LaTeX compiler: lualatex
\documentclass[11pt]{article}
\usepackage{graphicx}
\usepackage{grffile}
\usepackage{longtable}
\usepackage{wrapfig}
\usepackage{rotating}
\usepackage[normalem]{ulem}
\usepackage{amsmath}
\usepackage{textcomp}
\usepackage{amssymb}
\usepackage{capt-of}
\usepackage{hyperref}
\usepackage{tabularx}
\usepackage{etoolbox}
\makeatletter
\def\dontdofcolorbox{\renewcommand\fcolorbox[4][]{##4}}
\AtBeginEnvironment{minted}{\dontdofcolorbox}
\makeatother
\usepackage[newfloat]{minted}
\author{Mark Armstrong}
\date{September 25th, 2020}
\title{Computer Science 3MI3 – 2020 homework 3\\\medskip
\large Revisiting homework 1 in Prolog and homework 2 in Scala}
\hypersetup{
   pdfauthor={Mark Armstrong},
   pdftitle={Computer Science 3MI3 – 2020 homework 3},
   pdfkeywords={},
   pdfsubject={Reversing our introductions to Scala and Prolog.},
   pdfcreator={Emacs 27.0.90 (Org mode 9.4)},
   pdflang={English},
   colorlinks,
   linkcolor=blue,
   citecolor=blue,
   urlcolor=blue
   }
\begin{document}

\maketitle
\tableofcontents


\section*{Introduction}
\label{sec:orgd2a1ac1}
We have now worked in Scala for one homework,
representing different kinds of trees
and defining operations on them.

We have also worked in Prolog solving problems
about palindromes and prime numbers.

For our third homework, we gain further experience
in both languages by tackling our previous problems
in the other language.
Since they are from different paradigms,
although the problems are the same,
your solutions will out of necessity be quite different.

\section*{Boilerplate}
\label{sec:org92f059d}
\subsection*{Submission procedures}
\label{sec:orgf1d6350}
\subsubsection*{Submission method}
\label{sec:orgbc0b962}

Homework should be submitted to your McMaster CAS Gitlab respository
in the \texttt{cs3mi3-fall2020} project.

Ensure that you have \textbf{pushed} the commits to the remote repository
in time for the deadline, and not just committed to your local copy.

\subsubsection*{Naming requirements}
\label{sec:org68b9fa8}

Place all files for the homework
inside a folder titled \texttt{hn}, where \texttt{n} is the number of the homework.
So, for homework 1, the use the folder \texttt{h1}, for homework 2 the folder \texttt{h2}, etc.
Ensure you do not capitalise the \texttt{h}.

Unless otherwise instructed in the homework questions,
place all of your code for the homework
in a single file in the \texttt{hn} folder named \texttt{hn.ext},
where \texttt{ext} is the appropriate extension for the language used
according to this list:
\begin{itemize}
\item For Scala, \texttt{ext} is \texttt{sc}.
\item For Prolog, \texttt{ext} is \texttt{pl}.
\item For Ruby, \texttt{ext} is \texttt{rb}.
\item For Clojure, \texttt{ext} is \texttt{clg}.
\end{itemize}
If multiple languages are used in the homework,
submit a \texttt{hn.ext} file for each language.

\begin{center}
\textbf{If the language supports multiple different file extensions,}
\textbf{you must still follow the extension conventions above.}
\end{center}

\begin{center}
\textbf{Incorrect naming of files may result in up to a 10\% deduction in your grade.}
\end{center}

\subsubsection*{Do not submit testing or diagnostic code}
\label{sec:org347b3df}

Unless you are instructed to do so in the homework questions,
\textbf{you should not submit testing code with your homework submission}.

This includes
\begin{itemize}
\item any \texttt{main} function,
\item any \texttt{print} statements which output information
\textbf{that is not directly requested as console output}
\textbf{in the homework questions}.
\end{itemize}

If you do not wish to remove diagnostic print statements manually,
you will have to find a way to ensure that they disabled
in your final submission.

For instance, by using a wrapper on the print function or macros.

\subsubsection*{Due date and allowance for technical difficulties}
\label{sec:org69a30fa}

Homework is due on the second Sunday following its release,
by the end of the day (midnight).
Submissions past 00:00 may not be considered.

If you experience technical difficulties
leading up to the submission time,
please contact Mark \textbf{ASAP} with the details of the problem
and, if possible, attach the current state of your homework
to the communication.
This information will help ensure we are able
to accept your submission once the technical difficulties are resolved.

\subsection*{Proper conduct for coursework}
\label{sec:orgb2db236}
\subsubsection*{Individual work}
\label{sec:orge139aec}

Unless explicitely stated in the homework questions,
all homework in this course is intended
to be \emph{individually completed}.

You are welcome to discuss the content of the homework in
the public forum of the class Microsoft Teams team homework channel,
though obviously solutions or partial solutions should not
be posted or described.

Private discussions about the homework cannot reasonably be
forbidden, but such discussions should follow the same guidelines
as public discussions.

\begin{center}
\textbf{Inappopriate collaboration via private discussions}
\textbf{which is later discovered by course staff}
\textbf{may be considered academic dishonesty.}

When in doubt, make the discussion private, or report its contents
to the course staff by making a note of it
in your homework.
\end{center}

To clarify what is considered appropriate discussions
of homework content, here are some examples:
\begin{enumerate}
\item Discussing the language features introduced or needed for the homework.
\begin{itemize}
\item Such as relevant builtin datatypes
and datatype definition methods and their general use.
\item Code snippets that are not partial solutions to the homework
are welcome and encouraged.
\end{itemize}
\item Questions of the form “What is meant by \texttt{x}?”,
“Does \texttt{x} really mean \texttt{y}?” or “Is there a mistake with \texttt{x}?”
\begin{itemize}
\item Of course, questions of those form which would be answered
by partial solutions are not considered appropriate.
\end{itemize}
\item Questions or advice about errors that may be encountered.
\begin{itemize}
\item Such as “If you see a \texttt{scala.MatchError} you should
probably add a catch-all \texttt{\_} case to your \texttt{match} expressions.”
\end{itemize}
\end{enumerate}

\subsubsection*{Language library resources}
\label{sec:org8303005}

Unless explicitely stated in the questions,
it is not expected that you will use any language library resources
in the homeworks.

Possible exceptions to this rule include implementations
of datatypes we discuss in this course, such as lists
or options/maybes, if they are included in a standard library
instead of being builtin.

\emph{Basic} operations on such types would also be allowed.
\begin{itemize}
\item For instance, \texttt{head}, \texttt{tail}, \texttt{append}, etc. on lists
would not require explicit permission to be used.
\item More complex operations such as sorting procedures
would require permission before you used them.
\end{itemize}

Additionally, the standard \emph{higher-order} operations
including \texttt{map}, \texttt{reduce}, \texttt{flatten}, and \texttt{filter} are permitted generally,
unless the task is to implement such a higher-order operator.

\section*{Part 1: Flattening and ordering trees in Prolog [20 points]}
\label{sec:orgc362707}
Using the representations (not technically types)
of \texttt{BinTree} and \texttt{LeafTree} trees discussed in the
“Trees in Prolog” lecture (links below),
write predicates \texttt{flatten} and \texttt{elemsOrdered} in Prolog.
\begin{itemize}
\item (Each predicate is assigned 7.5 points.)
\end{itemize}

Each should be a binary predicate,
taking a tree and a list as parameters.
So \texttt{flatten(T,L)} means that \texttt{L} is the result of flattening \texttt{T},
and similarly \texttt{elemsOrdered(T,L)} means that \texttt{L} is a
list of elements of \texttt{T} in order.

Implement your own sorting predicate to use in \texttt{elemsOrdered};
do not use a predefined one.
\begin{itemize}
\item (The sorting predicate is assigned 5 points.)
\end{itemize}

Refer to the “Trees in Prolog” lecture notes, available as
\begin{itemize}
\item \href{./../notes/live-coding/09-25-Trees-in-Prolog.html}{HTML},
\item \href{./../notes/live-coding/09-25-Trees-in-Prolog.pdf}{PDF}, or
\item \href{./../notes/live-coding/09-25-Trees-in-Prolog.org}{plaintext Org source},
\end{itemize}

\section*{Part 2: Predicates in Scala                     [20 points]}
\label{sec:orgf20dc02}
In Scala and other non-logical languages,
we can represent predicates as boolean-valued functions/methods
(functions/methods which return booleans).
Note that in Scala the type of booleans is \texttt{Boolean}
(not \texttt{Bool}, as it is in some languages.)

In Scala, implement the predicates from homework 2:
\begin{enumerate}
\item \texttt{isPrime} on the \texttt{Int} type.
\begin{itemize}
\item (assigned 5 points.)
\end{itemize}
\item \texttt{isPalindrome} on the type \texttt{List[A]} for any \texttt{A}.
\begin{itemize}
\item (assigned 5 points.)
\end{itemize}
\item \texttt{primePalindrome} on the type \texttt{Int}.
\begin{itemize}
\item (assigned 5 points.)
\end{itemize}
\end{enumerate}

In the process, also define a method \texttt{digitList} (not \texttt{isDigitList})
which converts an \texttt{Int} into a \texttt{List[Int]} consisting
of its digits, with the first element of the list being
the “ones” digits, the second element being the “tens” digit, etc.
\begin{itemize}
\item (this method is assigned 5 points.)
\end{itemize}

\section*{Testing}
\label{sec:org7b222d0}
As of September 28th, unit tests for the requested predicates
are available through the following links.
\begin{itemize}
\item Scala unit tests: \href{./testing/h3/h3t.sc}{h3t.sc}
\item Prolog unit tests: \href{./testing/h3/h3.plt}{h3.plt}.
\end{itemize}
The contents of the unit test file are also repeated below.

The tests can be run by placing the \texttt{h3t.sc} and \texttt{h3.plt} files
in the same directory as your \texttt{h3.sc} and \texttt{h3.pl} files, and
running the following commands.
\begin{minted}[breaklines=true]{shell}
amm h3t.sc
\end{minted}
and
\begin{minted}[breaklines=true]{shell}
swipl -t "load_test_files([]), run_tests." -s h3.pl
\end{minted}
to test your Prolog code.

If you use implementations other than Ammonite and SWI Prolog,
you may need to adjust the tests to be able to run them
on your system.

In particular, those using IntelliJ for writing Scala
may need to remove the import statement at the top of that file,
or move the tests to their source file
(if so, be sure to remove them before submission!)

\subsection*{Automated testing via Docker}
\label{sec:orgeb1c87a}
:TODO:

\subsection*{The tests}
\label{sec:orgcd31717}
The contents of the testing scripts are repeated here.

\subsubsection*{The Scala tests}
\label{sec:org6281b1a}
\begin{minted}[breaklines=true]{scala}
import $file.h3, h3._

/* Given an expected result and a computed result,
   check if they are equal in value.
   If so, return 0. Otherwise, inform the user, and return 1,
   so the number of failures can be counted. */
def test[A](given: A, expected: A, the_test: String) =
  if (!(given equals expected)) {
    println("+---------------------------------------------------")
    println("| " + the_test + " failed.")
    println("| Expected " + expected + ", got " + given + ".")
    println("+---------------------------------------------------")
    1
  } else {
    0
  }

// The tests are saved as tuples, the pieces of which will be passed
// to test.
val tests = List(
  (isPrime(2), true, "Two is prime"),
  (isPrime(53), true, "Fifty three is prime"),
  (isPrime(8), false, "Eight is not prime"),
  (isPrime(63), false, "Sixty-three is not prime"),
  (isPalindrome(List()), true, "Empty palindrome"),
  (isPalindrome(List(1,1)), true, "Pair palindrome"),
  (isPalindrome(List('s','t','e','p','o','n','n','o','p','e','t','s')), true, "step on no pets palindrome"),
  (isPalindrome(List(12,21)), false, "Palindrome elements are atomic"),
  (primePalindrome(11), true, "Eleven is prime palindrome"),
  (primePalindrome(929), true, "Nine twenty nine is prime palindrome"),
  (primePalindrome(13), false, "Thirteen is not prime palindrome"),
  (primePalindrome(22), false, "Twenty two is not prime palindrome"),
  )

// Apply test to each element of tests, and sum the return values.
// This is essentially a for loop.
val failed = tests.foldLeft(0) {
  (failures, next) => next match {
    // Deconstruct the tuple to get its parts
    case (given, expected, the_test) => failures + test(given, expected, the_test)
  }
}

println("+---------------------------------------------------")
println("| " + failed + " tests failed")
println("+---------------------------------------------------")
\end{minted}

\subsubsection*{The Prolog tests}
\label{sec:orgc6e83ae}
\begin{minted}[breaklines=true]{prolog}
:- begin_tests(h3).
:- include(h3).

% One way to "name" values is to use predictes such as these.
% Then a variable can be bound to the values here using the predicate.
largeLeafTree(T) :-
    T = branch(
         branch(
          branch(leaf(10),leaf(-2)),
          branch(leaf(3),leaf(400))),
         branch(
          branch(leaf(55),leaf(6)),
          branch(leaf(777),leaf(88888888)))).
largeLeafTreeFlat(L) :-    L = [10,-2,3,400,55,6,777,88888888].
largeLeafTreeOrdered(L) :- L = [-2,3,6,10,55,400,777,88888888].

largeBinTree(T) :-
    T = node(
         node(
          node(empty,1,empty),
          22,
          node(empty,-3,empty)),
         404,     
         node(
          node(empty,50,node(empty,6,empty)),
          7777777,
          node(node(empty,-88,empty),9001,empty))).
largeBinTreeFlat(L) :-    L = [1,22,-3,404,50,6,7777777,-88,9001].
largeBinTreeOrdered(L) :- L = [-88,-3,1,6,22,50,404,9001,7777777].


% The tests

test(flatten_leaf, nondet) :- flatten(leaf(5),[5]).
test(flatten_branch, nondet) :- flatten(branch(leaf(1),leaf(2)),[1,2]).
test(flatten_large_leaftree, nondet) :- largeLeafTree(X), largeLeafTreeFlat(L), flatten(X,L).

test(order_leaf, nondet) :- elemsOrdered(leaf(5),[5]).
test(order_branch, nondet) :- elemsOrdered(branch(leaf(1),leaf(2)),[1,2]).
test(order_large_leaftree, nondet) :- largeLeafTree(X), largeLeafTreeOrdered(L), elemsOrdered(X,L).

test(flatten_empty, nondet) :- flatten(empty,[]).
test(flatten_node, nondet) :- flatten(node(empty,5,empty),[5]).
test(flatten_large_bintree, nondet) :- largeBinTree(X), largeBinTreeFlat(L), flatten(X,L).

test(order_empty, nondet) :- elemsOrdered(empty,[]).
test(order_node, nondet) :- elemsOrdered(node(empty,5,empty),[5]).
test(order_large_bintree, nondet) :- largeBinTree(X), largeBinTreeOrdered(L), elemsOrdered(X,L).
\end{minted}
\end{document}
