% Created 2020-10-23 Fri 23:51
% Intended LaTeX compiler: lualatex
\documentclass[11pt]{article}
\usepackage{graphicx}
\usepackage{grffile}
\usepackage{longtable}
\usepackage{wrapfig}
\usepackage{rotating}
\usepackage[normalem]{ulem}
\usepackage{amsmath}
\usepackage{textcomp}
\usepackage{amssymb}
\usepackage{capt-of}
\usepackage{hyperref}
\usepackage{tabularx}
\usepackage{etoolbox}
\makeatletter
\def\dontdofcolorbox{\renewcommand\fcolorbox[4][]{##4}}
\AtBeginEnvironment{minted}{\dontdofcolorbox}
\makeatother
\usepackage[newfloat]{minted}
\author{Mark Armstrong}
\date{October 23rd, 2020}
\title{Computer Science 3MI3 – 2020 homework 6\\\medskip
\large Representing and interpreting \texttt{Expr} in Ruby}
\hypersetup{
   pdfauthor={Mark Armstrong},
   pdftitle={Computer Science 3MI3 – 2020 homework 6},
   pdfkeywords={},
   pdfsubject={Repeating part 1 of the first assignment in Ruby.},
   pdfcreator={Emacs 27.0.90 (Org mode 9.4)},
   pdflang={English},
   colorlinks,
   linkcolor=blue,
   citecolor=blue,
   urlcolor=blue
   }
\begin{document}

\maketitle
\tableofcontents


\section*{Introduction}
\label{sec:orgc6ce153}
Recall the \texttt{Expr} language of integer expressions
using prefix operators from part 1 of assignment 1.

In this homework, we task you with representing this type
using an object-oriented approach in Ruby.

\section*{Boilerplate}
\label{sec:orgbce4191}
\subsection*{Submission procedures}
\label{sec:org5694653}
\subsubsection*{Submission method}
\label{sec:orgbafe198}

Homework should be submitted to your McMaster CAS Gitlab respository
in the \texttt{cs3mi3-fall2020} project.

Ensure that you have \textbf{pushed} the commits to the remote repository
in time for the deadline, and not just committed to your local copy.

\subsubsection*{Naming requirements}
\label{sec:org844b9d6}

Place all files for the homework
inside a folder titled \texttt{hn}, where \texttt{n} is the number of the homework.
So, for homework 1, use the folder \texttt{h1}, for homework 2 the folder \texttt{h2}, etc.
Ensure you do not capitalise the \texttt{h}.

Unless otherwise instructed in the homework questions,
place all of your code for the homework
in a single file in the \texttt{hn} folder named \texttt{hn.ext},
where \texttt{ext} is the appropriate extension for the language used
according to this list:
\begin{itemize}
\item For Scala, \texttt{ext} is \texttt{sc}.
\item For Prolog, \texttt{ext} is \texttt{pl}.
\item For Ruby, \texttt{ext} is \texttt{rb}.
\item For Clojure, \texttt{ext} is \texttt{clg}.
\end{itemize}
If multiple languages are used in the homework,
submit a \texttt{hn.ext} file for each language.

\begin{center}
\textbf{If the language supports multiple different file extensions,}
\textbf{you must still follow the extension conventions above.}
\end{center}

\begin{center}
\textbf{Incorrect naming of files may result in up to a 10\% deduction in your grade.}
\end{center}

\subsubsection*{Do not submit testing or diagnostic code}
\label{sec:org5f2eda1}

Unless you are instructed to do so in the homework questions,
\textbf{you should not submit testing code with your homework submission}.

This includes
\begin{itemize}
\item any \texttt{main} function,
\item any \texttt{print} statements which output information
\textbf{that is not directly requested as console output}
\textbf{in the homework questions}.
\end{itemize}

If you do not wish to remove diagnostic print statements manually,
you will have to find a way to ensure that they disabled
in your final submission.

For instance, by using a wrapper on the print function or macros.

\subsubsection*{Due date and allowance for technical difficulties}
\label{sec:org50850bd}

Homework is due on the second Sunday following its release,
by the end of the day (midnight).
Submissions past 00:00 may not be considered.

If you experience technical difficulties
leading up to the submission time,
please contact Mark \textbf{ASAP} with the details of the problem
and, if possible, attach the current state of your homework
to the communication.
This information will help ensure we are able
to accept your submission once the technical difficulties are resolved.

\subsection*{Proper conduct for coursework}
\label{sec:org6614d65}
\subsubsection*{Individual work}
\label{sec:orgab1544a}

Unless explicitely stated in the homework questions,
all homework in this course is intended
to be \emph{individually completed}.

You are welcome to discuss the content of the homework in
the public forum of the class Microsoft Teams team homework channel,
though obviously solutions or partial solutions should not
be posted or described.

Private discussions about the homework cannot reasonably be
forbidden, but such discussions should follow the same guidelines
as public discussions.

\begin{center}
\textbf{Inappopriate collaboration via private discussions}
\textbf{which is later discovered by course staff}
\textbf{may be considered academic dishonesty.}

When in doubt, make the discussion private, or report its contents
to the course staff by making a note of it
in your homework.
\end{center}

To clarify what is considered appropriate discussions
of homework content, here are some examples:
\begin{enumerate}
\item Discussing the language features introduced or needed for the homework.
\begin{itemize}
\item Such as relevant builtin datatypes
and datatype definition methods and their general use.
\item Code snippets that are not partial solutions to the homework
are welcome and encouraged.
\end{itemize}
\item Questions of the form “What is meant by \texttt{x}?”,
“Does \texttt{x} really mean \texttt{y}?” or “Is there a mistake with \texttt{x}?”
\begin{itemize}
\item Of course, questions of those form which would be answered
by partial solutions are not considered appropriate.
\end{itemize}
\item Questions or advice about errors that may be encountered.
\begin{itemize}
\item Such as “If you see a \texttt{scala.MatchError} you should
probably add a catch-all \texttt{\_} case to your \texttt{match} expressions.”
\end{itemize}
\end{enumerate}

\subsubsection*{Language library resources}
\label{sec:org2c1f0dd}

Unless explicitely stated in the questions,
it is not expected that you will use any language library resources
in the homeworks.

Possible exceptions to this rule include implementations
of datatypes we discuss in this course, such as lists
or options/maybes, if they are included in a standard library
instead of being builtin.

\emph{Basic} operations on such types would also be allowed.
\begin{itemize}
\item For instance, \texttt{head}, \texttt{tail}, \texttt{append}, etc. on lists
would not require explicit permission to be used.
\item More complex operations such as sorting procedures
would require permission before you used them.
\end{itemize}

Additionally, the standard \emph{higher-order} operations
including \texttt{map}, \texttt{reduce}, \texttt{flatten}, and \texttt{filter} are permitted generally,
unless the task is to implement such a higher-order operator.

\section*{Part 0: Example written in C++                   [0 points]}
\label{sec:org6055041}
There are at least two approaches to representing the \texttt{Expr} type
in an object-oriented style;
either by
\begin{enumerate}
\item using a single \texttt{Expr} class which uses fields to track,
for a given object of the class,
which kind of expression is the top-level expression, or by
\item using an \texttt{Expr} interface or class which has as a \emph{subclass} for
each kind of expression.
\end{enumerate}

\subsection*{Approach 1: A single class with fields}
\label{sec:orga7349cf}
We use standard output to print the expressions.
\begin{minted}[breaklines=true]{c++}
#include <stdio.h>
\end{minted}

There is not a basic exponent operator in C++;
we simply define one ourselves.
\begin{minted}[breaklines=true]{c++}
int pow(int x, int y){
  int r = 1;
  while (y-- > 0) {
    r *= x;
  }
  return r;
}
\end{minted}

Our class contains several types (\texttt{enum}'s, \texttt{union}'s and \texttt{struct}'s)
for internal use in tracking the kind of expression used.
In Ruby, most of this setup is not necessary due to
the dynamic type system.
The \texttt{UOp} and \texttt{BOp} enumerations are outside the class,
as they will be used as arguments to the constructors.
\begin{minted}[breaklines=true]{c++}
enum UOp { neg, abs };
enum BOp { plus, times, minus, exp };

class IExpr {

  enum Arity { constant, unary, binary };

  // An operator is either a UOp or a BOp.
  // Do be careful with such an untagged union!
  union Op {UOp uop; BOp bop;};

  // An IExpr2 is a pair of IExpr's.
  struct IExpr2 {IExpr* fst; IExpr* snd;};

  // The arguments are either:
  // - a single integer (in case of constant),
  // - a single IExpr (in case of unary operator), or
  // - two IExpr's (in case of binary operator).
  // Again, be careful with untagged union type!
  union Arg {int value; IExpr* subexpr; IExpr2 subexprs;};
\end{minted}

Now, each object has an arity, a (top-level) operator,
and an argument (which, remember from the definition of \texttt{Arg} just above,
might be an integer, and \texttt{IExpr}, or a pair of \texttt{IExpr}'s.)
\begin{minted}[breaklines=true]{c++}
  Arity ar;
  Op op;
  Arg arg;
\end{minted}

The only methods are the constructors and the \texttt{interpret} method.
\begin{minted}[breaklines=true]{c++}
  public:
    IExpr(int c);
    IExpr(UOp op, IExpr* e);
    IExpr(BOp op, IExpr* e1, IExpr* e2);
    int interpret();
};
\end{minted}

The constructors are fairly uninteresting.
\begin{minted}[breaklines=true]{c++}
IExpr::IExpr(int c) {
  ar = constant;
  arg.value = c;
}

IExpr::IExpr(UOp uop, IExpr* e) {
  ar = unary;
  op.uop = uop;
  arg.subexpr = e;
}

IExpr::IExpr(BOp bop, IExpr* e1, IExpr* e2) {
  ar = binary;
  op.bop = bop;
  arg.subexprs.fst = e1;
  arg.subexprs.snd = e2;
}
\end{minted}

The interpreter uses \texttt{switch} statements
on the arity and then on the top-level operator.
\begin{minted}[breaklines=true]{c++}
int IExpr::interpret() {
  switch(ar) {
    case constant : { return arg.value; }

    case unary : {
      // Single subexpression to evaluate
      int r = arg.subexpr->interpret();
      switch(op.uop) {
        case neg : { return - r; }
        case abs : { return r > -r ? r : -r; }
      }
    }

    case binary : {
      // Two subexpressions to evaluate
      int r1 = arg.subexprs.fst->interpret();
      int r2 = arg.subexprs.snd->interpret();
      switch(op.bop) {
        case plus  : { return r1 + r2; }
        case times : { return r1 * r2; }
        case minus : { return r1 - r2; }
        case exp   : { return pow(r1,r2); }
      }
    }
  }
}
\end{minted}

We include a main method with some example uses of the interpreter.
As usual, do not include such code with your submission!
\begin{minted}[breaklines=true]{c++}
int main() {
  IExpr five(5);
  IExpr six(6);
  IExpr e1(neg, &five);
  IExpr e2(exp, &five, &six);

  printf("neg 5 evaluates to %d\n", e1.interpret());
  printf("exp 5 6 evaluates to %d\n", e2.interpret());
  return 0;
}
\end{minted}

\subsection*{Approach 2: Subclasses for each kind of expression}
\label{sec:org8130fdc}
:TODO:

\section*{Part 1: A representation and interpreter in Ruby [40 points]}
\label{sec:orgc12ad1a}
In Ruby, create a class \texttt{Expr} to represent
the type of integer expressions from part 1 of assignment 1.

In order to ensure that your code is compatible
with the tests provided, please also
create methods \texttt{construct\_const}, \texttt{construct\_neg}, \texttt{construct\_abs},
\texttt{construct\_plus}, \texttt{construct\_times}, \texttt{construct\_minus}, and \texttt{construct\_exp} which,
given either an integer, \texttt{Expr}, or two \texttt{Expr} arguments as appropriate,
returns the corresponding expression.

(The body of these methods should simply be call to an \texttt{Expr} constructor;
requiring you implement these simply ensures the tests will be
compatible with whatever constructors you have.)

\section*{Part 2: Add variables and substitution           [10 bonus points]}
\label{sec:org020cb0d}
Repeat part 2 of the assignment,
adding variables and a substitution operator to the language
of expressions \texttt{Expr}, in Ruby.

Call your new type \texttt{VarExpr}, and submit both it code to test it
in a file \texttt{h5b.rb}.

\section*{Testing}
\label{sec:orgfa7449e}
Unit tests for the requested methods/functions
are available \href{./testing/h6/h6t.rb}{here}.
The contents of the unit test file are also repeated below.

The tests can be run by placing the \texttt{h6t.rb} file
in the same directory as your \texttt{h6.rb} file, and
running the following command.
\begin{minted}[breaklines=true]{shell}
ruby h6t.rb
\end{minted}
You may also use \texttt{irb h6t.rb}, which will echo the script
as it is run.

If you wish to add more tests yourself,
see the \href{https://en.wikibooks.org/wiki/Ruby\_Programming/Unit\_testing}{documentation}
for Ruby.

\subsection*{Automated testing via Docker}
\label{sec:org69b21b1}
The Docker setup and usage scripts are available at the following links.
Their contents are also repeated below.
\begin{itemize}
\item \href{./testing/h6/Dockerfile}{Dockerfile}
\item \href{./testing/h6/docker-compose.yml}{docker-compose.yml}
\item \href{./testing/h6/setup.sh}{setup.sh}
\item \href{./testing/h6/run.sh}{run.sh}
\end{itemize}
Place them into your \texttt{h6} directory where your \texttt{h6.rb} file
and the \texttt{h6t.rb} (linked to above) file exist,
then run \texttt{setup.sh} and \texttt{run.sh}.

You may also refer to the README
for this testing setup and those files
\href{https://github.com/armkeh/principles-of-programming-languages/tree/master/homework/testing/h5}{on the course GitHub repo}.

Note that the use of the \texttt{setup.sh} and \texttt{run.sh} scripts assumes
that you are in a \texttt{bash} like shell; if you are on Windows,
and not using WSL or WSL2, you may have
to run the commands contained in those scripts manually.

\subsection*{The tests}
\label{sec:org9e3ae21}
The contents of the testing script are repeated here.

\href{./testing/h6/h6t.rb}{h6t.rb}
\begin{minted}[breaklines=true]{ruby}
require_relative "h6"
require "test/unit"

class SimpleTests < Test::Unit::TestCase

  def test_constant_expression_zero
    e = construct_const(0)
    assert_equal(0, e.interpret)
  end

  def test_additive_expression
    e1 = construct_const(5)
    e2 = construct_const(4)
    e = construct_plus(e1,e2)
    assert_equal(9, e.interpret)
  end

  def test_exponential_expression
    e1 = construct_const(5)
    e2 = construct_const(4)
    e = construct_exp(e1,e2)
    assert_equal(625, e.interpret)
  end
end
\end{minted}

\subsection*{The Docker setup}
\label{sec:orgb65ea2b}
The contents of the Docker setup files and scripts are repeated here.

\href{./testing/h5/Dockerfile}{Dockerfile}
\begin{minted}[breaklines=true]{docker}
FROM ruby:2.7.2-buster

# Set the name of the maintainers
MAINTAINER Habib Ghaffari Hadigheh, Mark Armstrong <ghaffh1@mcmaster.ca, armstmp@mcmaster.ca>

# Set the working directory
WORKDIR /opt/h6
\end{minted}

\href{./testing/h5/docker-compose.yml}{docker-compose.yml}
\begin{minted}[breaklines=true]{yaml}
version: '2'
services:
  service:
    build: .
    image: 3mi3_h6_docker_image
    volumes:
      - .:/opt/h6
    container_name: 3mi3_h6_container
    command: ['ruby', 'h6t.rb']
\end{minted}

\href{./testing/h5/setup.sh}{setup.sh}
\begin{minted}[breaklines=true]{shell}
docker-compose build --force-rm
\end{minted}

\href{./testing/h5/run.sh}{run.sh}
\begin{minted}[breaklines=true]{shell}
# Run the container
docker-compose up --force-recreate
# Stop the container after finishing the test run
docker-compose stop -t 1
\end{minted}

\href{./testing/h5/README.md}{README.md}
\begin{minted}[breaklines=true]{text}
# Instructions for automated testing using Docker

We have already created a `Dockerfile` here which specifies
all the necessary packages, etc., for compiling and running your code.
You only need to follow the instructions below to see 
the results of unit tests designed to check your implementation.

## Setup
We use `docker-compose` and its configuration file to build the image.
Assuming you have `docker` and `docker-compose` installed,
simply execute
```shell script
./setup.sh
```
to generate the image.

## Prepare your code for the running the tests
You only need to place the `h5t.rb` unit test file and
the `run.sh` file in the same directory as your `h5.rb` source file.

## Running the tests
As with the build process, we have already put
the configuration needed for running the test inside `docker-compose.yml`.
Simply execute
```shell script
./run.sh
```
to run and see the results of the tests.
\end{minted}
\end{document}
