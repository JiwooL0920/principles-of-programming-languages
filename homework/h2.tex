% Created 2020-09-25 Fri 21:04
% Intended LaTeX compiler: lualatex
\documentclass[11pt]{article}
\usepackage{graphicx}
\usepackage{grffile}
\usepackage{longtable}
\usepackage{wrapfig}
\usepackage{rotating}
\usepackage[normalem]{ulem}
\usepackage{amsmath}
\usepackage{textcomp}
\usepackage{amssymb}
\usepackage{capt-of}
\usepackage{hyperref}
\usepackage{tabularx}
\usepackage{etoolbox}
\makeatletter
\def\dontdofcolorbox{\renewcommand\fcolorbox[4][]{##4}}
\AtBeginEnvironment{minted}{\dontdofcolorbox}
\makeatother
\usepackage[newfloat]{minted}
\author{Mark Armstrong}
\date{September 18th, 2020}
\title{Computer Science 3MI3 – 2020 homework 2\\\medskip
\large Solving problems in Prolog}
\hypersetup{
   pdfauthor={Mark Armstrong},
   pdftitle={Computer Science 3MI3 – 2020 homework 2},
   pdfkeywords={},
   pdfsubject={Introductory homework on Prolog and predicates},
   pdfcreator={Emacs 27.0.90 (Org mode 9.3.8)},
   pdflang={English},
   colorlinks,
   linkcolor=blue,
   citecolor=blue,
   urlcolor=blue
   }
\begin{document}

\maketitle
\tableofcontents


\section*{Introduction}
\label{sec:orgb507244}
Prolog is a \emph{logical programming} language,
in which the task of the programmer
is to precisely describe the possible solution space,
rather than an algorithm to generate the solution.
The language runtime then performs a search
to find a solution fitting the description.

\begin{center}
Put simply, the programmer describes the problem, not the solution.
\end{center}

\section*{Boilerplate}
\label{sec:org49f3cef}
\subsection*{Submission procedures}
\label{sec:org27077b9}
\subsubsection*{Submission method}
\label{sec:org1e01d96}

Homework should be submitted to your McMaster CAS Gitlab respository
in the \texttt{cs3mi3-fall2020} project.

Ensure that you have \textbf{pushed} the commits to the remote repository
in time for the deadline, and not just committed to your local copy.

\subsubsection*{Naming requirements}
\label{sec:org27a7629}

Place all files for the homework
inside a folder titled \texttt{hn}, where \texttt{n} is the number of the homework.
So, for homework 1, the use the folder \texttt{h1}, for homework 2 the folder \texttt{h2}, etc.
Ensure you do not capitalise the \texttt{h}.

Unless otherwise instructed in the homework questions,
place all of your code for the homework
in a single file in the \texttt{hn} folder named \texttt{hn.ext},
where \texttt{ext} is the appropriate extension for the language used
according to this list:
\begin{itemize}
\item For Scala, \texttt{ext} is \texttt{sc}.
\item For Prolog, \texttt{ext} is \texttt{pl}.
\item For Ruby, \texttt{ext} is \texttt{rb}.
\item For Clojure, \texttt{ext} is \texttt{clg}.
\end{itemize}
If multiple languages are used in the homework,
submit a \texttt{hn.ext} file for each language.

\begin{center}
\textbf{If the language supports multiple different file extensions,}
\textbf{you must still follow the extension conventions above.}
\end{center}

\begin{center}
\textbf{Incorrect naming of files may result in up to a 10\% deduction in your grade.}
\end{center}

\subsubsection*{Do not submit testing or diagnostic code}
\label{sec:org4440178}

Unless you are instructed to do so in the homework questions,
\textbf{you should not submit testing code with your homework submission}.

This includes
\begin{itemize}
\item any \texttt{main} function,
\item any \texttt{print} statements which output information
\textbf{that is not directly requested as console output}
\textbf{in the homework questions}.
\end{itemize}

If you do not wish to remove diagnostic print statements manually,
you will have to find a way to ensure that they disabled
in your final submission.

For instance, by using a wrapper on the print function or macros.

\subsubsection*{Due date and allowance for technical difficulties}
\label{sec:org79b9a92}

Homework is due on the second Sunday following its release,
by the end of the day (midnight).
Submissions past 00:00 may not be considered.

If you experience technical difficulties
leading up to the submission time,
please contact Mark \textbf{ASAP} with the details of the problem
and, if possible, attach the current state of your homework
to the communication.
This information will help ensure we are able
to accept your submission once the technical difficulties are resolved.

\subsection*{Proper conduct for coursework}
\label{sec:orgdab1c91}
\subsubsection*{Individual work}
\label{sec:org18dd1ef}

Unless explicitely stated in the homework questions,
all homework in this course is intended
to be \emph{individually completed}.

You are welcome to discuss the content of the homework in
the public forum of the class Microsoft Teams team homework channel,
though obviously solutions or partial solutions should not
be posted or described.

Private discussions about the homework cannot reasonably be
forbidden, but such discussions should follow the same guidelines
as public discussions.

\begin{center}
\textbf{Inappopriate collaboration via private discussions}
\textbf{which is later discovered by course staff}
\textbf{may be considered academic dishonesty.}

When in doubt, make the discussion private, or report its contents
to the course staff by making a note of it
in your homework.
\end{center}

To clarify what is considered appropriate discussions
of homework content, here are some examples:
\begin{enumerate}
\item Discussing the language features introduced or needed for the homework.
\begin{itemize}
\item Such as relevant builtin datatypes
and datatype definition methods and their general use.
\item Code snippets that are not partial solutions to the homework
are welcome and encouraged.
\end{itemize}
\item Questions of the form “What is meant by \texttt{x}?”,
“Does \texttt{x} really mean \texttt{y}?” or “Is there a mistake with \texttt{x}?”
\begin{itemize}
\item Of course, questions of those form which would be answered
by partial solutions are not considered appropriate.
\end{itemize}
\item Questions or advice about errors that may be encountered.
\begin{itemize}
\item Such as “If you see a \texttt{scala.MatchError} you should
probably add a catch-all \texttt{\_} case to your \texttt{match} expressions.”
\end{itemize}
\end{enumerate}

\subsubsection*{Language library resources}
\label{sec:org70e34ef}

Unless explicitely stated in the questions,
it is not expected that you will use any language library resources
in the homeworks.

Possible exceptions to this rule include implementations
of datatypes we discuss in this course, such as lists
or options/maybes, if they are included in a standard library
instead of being builtin.

\emph{Basic} operations on such types would also be allowed.
\begin{itemize}
\item For instance, \texttt{head}, \texttt{tail}, \texttt{append}, etc. on lists
would not require explicit permission to be used.
\item More complex operations such as sorting procedures
would require permission before you used them.
\end{itemize}

Additionally, the standard \emph{higher-order} operations
including \texttt{map}, \texttt{reduce}, \texttt{flatten}, and \texttt{filter} are permitted generally,
unless the task is to implement such a higher-order operator.

\section*{Part 0.1: Installing Prolog                            [0 points]}
\label{sec:orgcf54e80}
In this course, we will be targetting
\href{https://www.swi-prolog.org/}{SWI Prolog} version 8.2.0,
as used in the
\href{https://hub.docker.com/\_/swipl/}{swipl}
Docker image.

\begin{center}
\textbf{If there is any update to the Docker image,}
\textbf{or if for any other reason we change our targeted versions},
\textbf{we will make an announcement on the homepage.}
\end{center}

That said, any recent version should suffice. 

\subsection*{Installation guides}
\label{sec:org9e8184d}
\begin{itemize}
\item Linux and MacOS users should be able to install SWI Prolog
via their package manager.
\item The \href{https://www.swi-prolog.org/download/stable}{SWI Prolog website}
provides downloads of prebuilt binaries for SWI Prolog 8.2.1.
\end{itemize}

\section*{Part 0.2: Basic Prolog programming                     [0 points]}
\label{sec:orgf2480c3}
Some basic tutorial on Scala will be given
in an upcoming lecture, and also in the tutorials,
and should provide you with the knowledge you need
to complete this homework.

\section*{Prolog specific instructions}
\label{sec:orgcb1eeb4}
In your Prolog code in this course,
you are allowed to use all builtin predicates;
in particular, see the following entries of the
SWI Prolog documentation:
\begin{itemize}
\item \href{https://www.swi-prolog.org/pldoc/man?section=arith}{Arithmetic} and
\item \href{https://www.swi-prolog.org/pldoc/man?section=builtinlist}{Built-in list operations}
\end{itemize}

\href{https://github.com/alhassy/PrologCheatSheet}{Musa AlHassy's Prolog cheat sheet}
(developed in part for this course last year.)
is also a useful resource, and
includes links to several other resources.

\section*{Part 1: Prime checker                                  [5 points]}
\label{sec:org005d9da}
Here we define a predicate “has divisor less than or equal to”
which checks, for a given \texttt{X}, if there exists a divisor
of \texttt{X} less than \textbf{or equal to} the given \texttt{Y} which is greater than 1.
(Edited September 20th: previously the preceding description
incorrectly said only “less than”.)
\begin{minted}[breaklines=true]{prolog}
hasDivisorLessThanOrEqualTo(_,1) :- !, false.
hasDivisorLessThanOrEqualTo(X,Y) :- 0 is X mod Y, !.
hasDivisorLessThanOrEqualTo(X,Y) :- Z is Y - 1, hasDivisorLessThanOrEqualTo(X,Z).
\end{minted}
Roughly, this works as follows:
\begin{enumerate}
\item If \texttt{Y} is 1, end the search (don't backtrack to look for other proofs)
and fail (search turns up \texttt{false}).
\item Otherwise, if \texttt{Y} does divide \texttt{X}, end the search immediately
(don't backtrack looking for other proofs)
(and the search will return \texttt{true}).
\item Otherwise, check if there is a divisor of \texttt{X} which is less than \texttt{Y - 1}.
\end{enumerate}

Using this predicate, define a \texttt{isPrime} predicate
such that \texttt{isPrime(X)} returns true if \texttt{X} is a prime number.

\section*{Part 2: From number to list of digits (and vice versa) [15 points]}
\label{sec:org8eb9022}
Complete the following definition of predicate that checks
if a given list is a list of the digits in a given integer,
with the first element of the list being the “ones” digit,
the second element of the list being the “tens” digit, etc.

\begin{center}
\textbf{Updated September 23rd: the given code below has been changed,}
\textbf{and the description of the list has been “reversed”.}
\textbf{Previously, the ones digits would have been the last element.}
\textbf{Now, it is the first element.}

One further edit was added later; remove the cut in the first case below.
\end{center}

We have filled in code to fetch the last element of the list
as a starting point.
\begin{minted}[breaklines=true]{prolog}
isDigitList(_,[]) :- false.
isDigitList(X,[X]). % Your code here; change the . to :-
isDigitList(X,[H|T]). % Your code here; change the . to :-
\end{minted}

(Note this predicate can be thought of as checking if the given list contains
the digits for the given integer, or as checking if the given integer
has the digits contained in a given list; hence, the “vice versa”.)

You may use this code to “drop the last element” of the list
if you wish (in this part and the rest of the homework.)
Updated September 23rd: this predicate is unlikely to be necessary
for the updated version of this question.
\begin{minted}[breaklines=true]{prolog}
% dropLast(L1,L2) if L2 is the list L1, leaving off the last item.
dropLast([_],[]). % The last element is dropped.
dropLast([H|T],[H|T2]) :-
  % Aside from the base case above, the lists must match.
  dropLast(T,T2).
\end{minted}

\section*{Part 3: Palindrome                                     [10 points]}
\label{sec:orgfde2f49}
Define a predicate \texttt{isPalindrome} which checks that a given
list is a palindrome.

Note you can use pattern matching with the list “cons” operator, \texttt{|},
to refer to the head and tail of a list, as in
\begin{verbatim}
palindrome([H|T]) :- ??? % H is the first element, T is the rest
\end{verbatim}

\section*{Part 4: Prime palindromes                              [15 points]}
\label{sec:org0771564}
Define a predicate \texttt{primePalindrome} which checks
if a given integer is both
\begin{enumerate}
\item prime, and
\item a palindrome (in terms of the list of its digits).
\end{enumerate}

\section*{Part 5: Efficiency                                     [10 bonus points]}
\label{sec:org891d4ee}
For extra marks, research how to make your above definitions more efficient.

In particular, look into the use of the “cut” operator, \texttt{!},
to prevent backtracking.

Updated September 25th: these test scripts,
separate from the provided unit testing, can be used
to evaluate the efficiency of your solution.
On my system (an admittedly underpowered Chromebook),
the \href{./testing/h2/h2-timing.pl}{first test} takes roughly 46 seconds.
The \href{./testing/h2/h2-beastly-timing.pl}{second test} takes several minutes.
The \href{./testing/h2/h2-belphegor-timing.pl}{last one} did not halt before I gave up on it 😀.
\begin{minted}[breaklines=true]{prolog}
:- consult(h2).

:- initialization(writeln('Timing prime palindrome')).
:- initialization(time(primePalindrome(111010111))).
\end{minted}
\begin{minted}[breaklines=true]{prolog}
:- consult(h2).

:- initialization(writeln('Timing prime palindrome with a beastly prime palindrome')).
:- initialization(time(primePalindrome(700666007))).
\end{minted}
\begin{minted}[breaklines=true]{prolog}
:- consult(h2).

:- initialization(writeln('Timing prime palindrome with Belphegor\'s prime palindrome')).
:- initialization(time(primePalindrome(1000000000000066600000000000001))).
\end{minted}

\section*{Testing}
\label{sec:org8e52d6b}
As of September 23rd, unit tests for the requested predicates
can be found \href{./testing/h2/h2.plt}{here}. The contents of the unit test file are also repeated below.

Important notes regarding this testing:
\begin{enumerate}
\item Unlike homework 1, for homework 2 it is \emph{mandatory}
that your code pass these tests.
\item There is no interface file that you need to complete.
You still only need submit your \texttt{h2.pl} file.
\item Passing these tests does not guarantee a particular grade.
What has been provided is intended to act as a \emph{sanity check},
and more thorough tests may be used in the grading process.
\end{enumerate}

The tests can be run by placing the \texttt{h2.plt} unit test file
in the same directory as your \texttt{h2.pl}, and running the command
\begin{minted}[breaklines=true]{shell}
swipl -t "load_test_files([]), run_tests." -s h2.pl
\end{minted}
in your shell in that directory.

\subsection*{Automated testing via Docker}
\label{sec:org58ac8a6}
Once you have \href{https://docs.docker.com/get-docker/}{Docker}
and \href{https://docs.docker.com/compose/install/}{Docker Compose}
(installation guides linked here)
installed, you can easily run the unit tests
by copying the files
\begin{itemize}
\item \texttt{Dockerfile},
\item \texttt{docker-compose.yml},
\item \texttt{setup.sh} and
\item \texttt{run.sh}
\end{itemize}
into your \texttt{h2} directory where your \texttt{h2.pl} file exists,
and running \texttt{setup.sh} and \texttt{run.sh}.

See the README for this testing setup and those files
\href{https://github.com/armkeh/principles-of-programming-languages/tree/master/homework/testing/h2}{on the course GitHub repo}.

Note that the use of the \texttt{setup.sh} and \texttt{run.sh} scripts assumes
that you are in a \texttt{bash} like shell; if you are on Windows,
and not using WSL or WSL2, you may have
to run the commands contained in those scripts manually.

\subsection*{The tests}
\label{sec:orgb7fffb3}
The contents of
\href{./testing/h2/h2.plt}{the testing script}
are repeated here.
\begin{minted}[breaklines=true]{prolog}
:- begin_tests(h2).
:- include(h2).


% isPrime tests

test(five_is_prime, nondet) :- isPrime(5).
test(sixty_seven_is_prime, nondet) :- isPrime(67).
% The \+ means “it is not provable that.”
test(four_is_not_prime, nondet) :- \+ isPrime(4).
test(one_hundred_is_not_prime, nondet) :- \+ isPrime(100).
% Also test there are no “bad” proofs that primes are not prime.
test(not_five_is_not_prime, nondet) :- \+ not(isPrime(5)).



% isDigitList tests

test(digit_list_12345, nondet) :- isDigitList(12345,[5,4,3,2,1]).
test(not_digit_list_12345_grouped, nondet) :- \+ isDigitList(12345,[5,43,2,1]).



% isPalindrome tests

test(empty_palindrome, nondet) :- isPalindrome([]).
test(pair_palindrome, nondet) :- isPalindrome([1,1]).
test(steponnopets_palindrome, nondet) :- isPalindrome([s,t,e,p,o,n,n,o,p,e,t,s]).
test(palindrome_elements_atomic, nondet) :- \+ isPalindrome([12, 21]).



% primePalindrome tests

test(eleven_prime_palindrome, nondet) :- primePalindrome(11).
test(nine_twenty_nine_prime_palindrome, nondet) :- primePalindrome(929).
test(not_thirteen_prime_palindrome, nondet) :- \+ primePalindrome(13).
test(not_twenty_two_prime_palindrome, nondet) :- \+ primePalindrome(22).
\end{minted}
\end{document}
